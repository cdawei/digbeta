\section{Cast playlist generation as multi-label classification: a formalised example}

Suppose we have 
2 playlists $\{(\textit{user}_1, [\textit{song}_1, \textit{song}_3]), (\textit{user}_2, [\textit{song}_2, \textit{song}_3])\}$
which include 
3 songs $\{\textit{song}_1, \textit{song}_2, \textit{song}_3\}$,
and 2 users $\{\textit{user}_1, \textit{user}_2\}$.

The audio features of songs are shown in Table~\ref{tab:audio}, by factorising the user-song matrix (of ratings), 
we can learn latent features for users (Table~\ref{tab:user-song}) and songs (Table~\ref{tab:song-user});
similarly, factorising the artist-song matrix (of ratings) can generate latent features of artists and songs (Table~\ref{tab:song-artist}).

\vspace{1em}
\begin{minipage}[t][3em][b]{.35\linewidth}
\begin{tabular}{l|ccc} \hline \hline 
& feature$_1$ & feature$_2$ & feature$_3$ \\ \hline 
{\it song}$_1$ & $a_{11}$ & $a_{12}$ & $a_{13}$ \\
{\it song}$_2$ & $a_{21}$ & $a_{22}$ & $a_{23}$ \\
{\it song}$_3$ & $a_{31}$ & $a_{32}$ & $a_{33}$ \\ \hline
\end{tabular}
\captionof{table}{Audio features of songs}
\label{tab:audio}
\end{minipage}
%
\quad
%
\begin{minipage}{.31\linewidth}
\begin{tabular}{l|cc} \hline \hline 
& feature$_1$ & feature$_2$ \\ \hline
{\it song}$_1$ & $su_{11}$ & $su_{12}$ \\
{\it song}$_2$ & $su_{21}$ & $su_{22}$ \\
{\it song}$_3$ & $su_{31}$ & $su_{32}$ \\ \hline
\end{tabular}
\captionof{table}{Latent song features learnt by factorising user-song matrix}
\label{tab:song-user}
\end{minipage}
%
\quad
%
\begin{minipage}{.31\linewidth}
\begin{tabular}{l|cc} \hline \hline 
& feature$_1$ & feature$_2$ \\ \hline 
{\it song}$_1$ & $sa_{11}$ & $sa_{12}$ \\
{\it song}$_2$ & $sa_{21}$ & $sa_{22}$ \\
{\it song}$_3$ & $sa_{31}$ & $sa_{32}$ \\ \hline
\end{tabular}
\captionof{table}{Latent song features learnt by factorising artist-song matrix}
\label{tab:song-artist}
\end{minipage}


\begin{table}[h!]
\centering
\begin{tabular}{l|cc} \hline \hline 
& feature$_1$ & feature$_2$ \\ \hline
{\it user}$_1$ & $u_{11}$ & $u_{12}$ \\
{\it user}$_2$ & $u_{21}$ & $u_{22}$ \\ \hline
\end{tabular}
\caption{Latent user features learnt by factorising user-song matrix}
\label{tab:user-song}
\end{table}

We concatenate the audio song features, 
latent song features learnt by factorising user-song matrix,
and latent song features learnt by factorising artist-song matrix 
to get the matrix of song features, as shown in Table~\ref{tab:song}.

\begin{table}[h!]
\centering
\begin{tabular}{l|ccc|cc|cc} \hline \hline 
& feature$_1$ & feature$_2$ & feature$_3$ & feature$_4$ & feature$_5$ & feature$_6$ & feature$_7$ \\ \hline
{\it song}$_1$ & $a_{11}$ & $a_{12}$ & $a_{13}$ & $su_{11}$ & $su_{12}$ & $sa_{11}$ & $sa_{12}$ \\          
{\it song}$_2$ & $a_{21}$ & $a_{22}$ & $a_{23}$ & $su_{21}$ & $su_{22}$ & $sa_{21}$ & $sa_{22}$ \\          
{\it song}$_3$ & $a_{31}$ & $a_{32}$ & $a_{33}$ & $su_{31}$ & $su_{32}$ & $sa_{31}$ & $sa_{32}$ \\ \hline
\end{tabular}
\caption{Features of songs}
\label{tab:song}
\end{table}

We assume a query is a tuple of $(\textit{user}, \textit{seed\_song})$.
For playlist $\{(\textit{user}_1, [\textit{song}_1, \textit{song}_3])$,
we have a query $\textit{query}_1 = (\textit{user}_1, \textit{song}_1)$;
for playlist $(\textit{user}_2, [\textit{song}_2, \textit{song}_3])\}$,
we have a query $\textit{query}_2 = (\textit{user}_2, \textit{song}_2)$;

We compute the song-query feature matrix $\mathbf{X}$ for the training set (with only 2 playlists) 
by concatenating the features of user in query, features of seed song in query, as shown in Table~\ref{tab:trainx}.

\begin{table}[h!]
\centering
\begin{tabular}{l|cc|ccccccc} \hline \hline
& feature$_1$ & feature$_2$ & feature$_3$ & feature$_4$ & feature$_5$ & feature$_6$ & feature$_7$ & feature$_8$ & feature$_9$ \\ \hline
$\x\pb{1} = \textit{query}_1$
& $u_{11}$ & $u_{12}$ 
& $a_{11}$ & $a_{12}$ & $a_{13}$ & $su_{11}$ & $su_{12}$ & $sa_{11}$ & $sa_{12}$ \\
$\x\pb{2} = \textit{query}_2$ 
& $u_{21}$ & $u_{22}$ 
& $a_{21}$ & $a_{22}$ & $a_{23}$ & $su_{21}$ & $su_{22}$ & $sa_{21}$ & $sa_{22}$ \\ \hline
\end{tabular}
\caption{Query-song feature matrix $\mathbf{X}$}
\label{tab:trainx}
\end{table}

We have a label for each song in the music library, and the labels of these two queries are shown in Table~\ref{tab:trainy}.

\begin{table}[h!]
\centering
\begin{tabular}{l|ccc} \hline \hline
& Label$_1$ & Label$_2$ & Label$_3$ \\ \hline
$\y\pb{1}$ & $1$ & $0$ & $1$ \\
$\y\pb{2}$ & $0$ & $1$ & $1$ \\ \hline
\end{tabular}
\caption{Labels of playlists}
\label{tab:trainy}
\end{table}

Do we need joint features $\Psi(\x, \y)$? Then it will become another structured prediction problem.


\eat{
\begin{table}
\begin{tabular}{lccccccccc|ccccccc} \hline \hline \\
& feature$_1$ & feature$_2$ & feature$_3$ & feature$_4$ & feature$_5$ & feature$_6$ & feature$_7$ \hline \\
(\textit{query}$_1$, \textit{song}$_1$) 
& $u_{11}$ & $u_{12}$ 
& $a_{11}$ & $a_{12}$ & $a_{13}$ & $su_{11}$ & $su_{12}$ & $sa_{11}$ & $sa_{12}$ 
& $a_{11}$ & $a_{12}$ & $a_{13}$ & $su_{11}$ & $su_{12}$ & $sa_{11}$ & $sa_{12}$ \\
(\textit{query}$_1$, \textit{song}$_2$) 
& $u_{11}$ & $u_{12}$ 
& $a_{11}$ & $a_{12}$ & $a_{13}$ & $su_{11}$ & $su_{12}$ & $sa_{11}$ & $sa_{12}$ 
& $a_{21}$ & $a_{22}$ & $a_{23}$ & $su_{21}$ & $su_{22}$ & $sa_{21}$ & $sa_{22}$ \\          
(\textit{query}$_1$, \textit{song}$_3$) 
& $u_{11}$ & $u_{12}$ 
& $a_{11}$ & $a_{12}$ & $a_{13}$ & $su_{11}$ & $su_{12}$ & $sa_{11}$ & $sa_{12}$ 
& $a_{31}$ & $a_{32}$ & $a_{33}$ & $su_{31}$ & $su_{32}$ & $sa_{31}$ & $sa_{32}$ \\ \hline
\end{tabular}
\caption{Query-song feature matrix}
\end{table}
}
