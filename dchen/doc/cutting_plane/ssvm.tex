\section{Training structured SVM using cutting-plane methods}
\label{sec:ssvm_train}

\subsection{Training the $n$-slack formulation of structured SVM}
\label{sec:nslackssvm}

Given $n$ training examples $(\mathbf{x}_1, \mathbf{y}_1), \dots, (\mathbf{x}_n, \mathbf{y}_n)$, 
the structured SVM with margin-rescaling\footnote{For brevity, structured SVM (SSVM) with slack-rescaling are not described in this document.}
can be formulated as a quadratic program (QP)
\begin{equation}
\label{eq:nslackform}
\begin{aligned}
\min_{\mathbf{w}, ~\bm{\xi} \ge 0} ~& \frac{1}{2} \mathbf{w}^\top \mathbf{w} + \frac{C}{n} \sum_{i=1}^n \xi_i \\
s.t.~~ ~& \mathbf{w}^\top \Psi(\mathbf{x}_i, \mathbf{y}_i) - \mathbf{w}^\top \Psi(\mathbf{x}_i, \bar{\mathbf{y}}) \ge 
       \Delta(\mathbf{y}_i, \bar{\mathbf{y}}) - \xi_i, ~(\forall i,~ \bar{\mathbf{y}} \neq \mathbf{y}_i)
\end{aligned}
\end{equation}
where $\mathbf{w}$ is the parameter vector, $C > 0$ is a regularisation constant,
$\Delta(\mathbf{y}, \bar{\mathbf{y}})$ is a discrepancy function that measures the loss 
for predicting $\bar{\mathbf{y}}$ given ground truth $\mathbf{y}$.
and $\xi_i$ is a slack variable that represents the \emph{hinge loss} associated with 
the prediction for the $i$-th example~\cite{tsochantaridis2005large},
\begin{equation}
\label{eq:nslackloss}
\xi_i = \max \left( 0,~ 
        \max_{\bar{\mathbf{y}} \in \mathcal{Y}} 
        \left\{ \Delta(\mathbf{y}_i, \bar{\mathbf{y}}) + \mathbf{w}^\top \Psi(\mathbf{x}_i, \bar{\mathbf{y}}) \right\} -
        \mathbf{w}^\top \Psi(\mathbf{x}_i, \mathbf{y}_i) \right).
\end{equation}
This formulation is called "$n$-slack" as we have one slack variable for each example in training set. \eat{citation}

To train the $n$-slack formulation of structured SVM, one option is simply enumerating all constraints and 
solve optimisation problem (\ref{eq:nslackform}) using a standard QP solver, 
however, this approach is impractical as there is a constraint for every incorrect label $\bar{\mathbf{y}}$.
Instead, we use a cutting-plane algorithm that repeatedly solves QP (\ref{eq:nslackform}) with respect to different set of constraints, 
and each iteration generates a new constraint that helps reduce the feasible region of the problem, 
until a specified precision $\varepsilon$ is achieved~\cite{joachims2009predicting}, as described in Algorithm~\ref{alg:nslacktrain}.

\begin{algorithm}[htbp]
\caption{Cutting-plane algorithm for training $n$-slack formulation of structured SVM (with margin-rescaling)}
\label{alg:nslacktrain}
\begin{algorithmic}[1]
\STATE \textbf{Input}: $\left( (\mathbf{x}_1, \mathbf{y}_1), \dots, (\mathbf{x}_n, \mathbf{y}_n) \right),~ C,~ \varepsilon$
\STATE $\mathcal{W} = \emptyset,~\mathcal{S}_i = \emptyset,~ k = 1,~ \mathbf{w}^{(k)} = \mathbf{0},~ \bm{\xi}^{(k)} = \mathbf{0}$
\REPEAT
    \FOR{$i = 1,\dots,n$}
        \STATE $\triangleright$ Query the oracle at point $q^{(k)} = (\mathbf{w}^{(k)}, \bm{\xi}^{(k)})$ as follows
        \STATE Do loss-augmented inference:~
               $\hat{\mathbf{y}}^{(k)} = \argmax_{\bar{\mathbf{y}} \in \mathcal{Y}} \{ \Delta(\mathbf{y}_i, \bar{\mathbf{y}}) + 
                \langle \mathbf{w}^{(k)},~ \Psi(\mathbf{x}_i, \bar{\mathbf{y}}) \rangle \}$ 
        \IF{~$q^{(k)}$ is not $\varepsilon$-feasible:~ 
             $\langle \mathbf{w}^{(k)},~ \Psi(\mathbf{x}_i, \mathbf{y}_i) - \Psi(\mathbf{x}_i, \hat{\mathbf{y}}^{(k)}) \rangle + 
             \varepsilon < \Delta(\mathbf{y}_i, \hat{\mathbf{y}}^{(k)}) - \xi_i^{(k)}$~}
            \STATE $\triangleright$ Form a \emph{feasibility cut} and update constraints
            \STATE $\mathcal{W} = \mathcal{W} \cup 
                    \left\{ \langle \mathbf{w},~ \Psi(\mathbf{x}_i, \mathbf{y}_i) - \Psi(\mathbf{x}_i, \hat{\mathbf{y}}^{(k)}) \rangle \ge 
                    \Delta(\mathbf{y}_i, \hat{\mathbf{y}}^{(k)}) - \xi_i \right\},~ \mathcal{S}_i = \mathcal{S}_i \cup \{\hat{\mathbf{y}}^{(k)} \}$ 
            \STATE Generate the next query point $q^{(k+1)} = (\mathbf{w}^{(k+1)}, \bm{\xi}^{(k+1)})$ 
                   by solving QP~(\ref{eq:nslackform}) w.r.t. all constraints in $\mathcal{W}$
            \STATE $k = k+1$
        \ENDIF
    \ENDFOR
%\UNTIL{$\mathcal{W}$ has not changed during iteration}
\UNTIL{$q^{(k)}$ is $\varepsilon$-feasible for all training examples}
\RETURN $q^{(k)}$
\end{algorithmic}
\end{algorithm}

Alternatively, the query point in Algorithm~\ref{alg:nslacktrain} will contain only $\mathbf{w}^{(k)}$
if we compute the loss $\xi_i^{(k)}$ on the fly~\cite{tsochantaridis2004support}
\begin{equation*}
\xi_i^{(k)} = \max \left( 0,~ 
              \max_{\bar{\mathbf{y}} \in \mathcal{S}_i} 
              \left\{ \Delta(\mathbf{y}_i, \bar{\mathbf{y}}) + \langle \mathbf{w}^{(k)}, \Psi(\mathbf{x}_i, \bar{\mathbf{y}}) \rangle \right\} -
              \langle \mathbf{w}^{(k)}, \Psi(\mathbf{x}_i, \mathbf{y}_i) \rangle \right).
\end{equation*}


\subsection{Training the $1$-slack formulation of structured SVM}
\label{sec:1slackssvm}

Another formulation of structured SVM which results in more efficient training is called "$1$-slack" formulation (with margin-rescaling),
it replaces the $n$ cutting-plane models of the hinge loss (one for each training example) with a single cutting-plane model for 
the sum of the hinge-losses~\cite{joachims2009cutting}, as a result, only one slack variable is needed,
\begin{equation}
\label{eq:1slackform}
\begin{aligned}
\min_{\mathbf{w}, ~\xi \ge 0} ~& \frac{1}{2} \mathbf{w}^\top \mathbf{w} + C \xi \\
s.t.~~ ~& \forall(\bar{\mathbf{y}}_1, \dots, \bar{\mathbf{y}}_n) \in \mathcal{Y}^n: 
          \frac{1}{n} \sum_{i=1}^n 
          \left( \mathbf{w}^\top \Psi(\mathbf{x}_i, \mathbf{y}_i) - \mathbf{w}^\top \Psi(\mathbf{x}_i, \bar{\mathbf{y}}_i) \right) \ge
          \frac{1}{n} \sum_{i=1}^n \Delta(\mathbf{y}_i, \bar{\mathbf{y}}_i) - \xi.
\end{aligned}
\end{equation}
Here the slack variable $\xi$ represents the \emph{sum of the hinge-losses} over all training examples,
\begin{equation}
\label{eq:1slackloss}
\xi = \max \left( 0,~ 
      \max_{(\bar{\mathbf{y}}_1, \dots, \bar{\mathbf{y}}_n) \in \mathcal{Y}^n} 
      \left\{ 
      \frac{1}{n} \sum_{i=1}^n \left( \Delta(\mathbf{y}_i, \bar{\mathbf{y}}_i) + \mathbf{w}^\top \Psi(\mathbf{x}_i, \bar{\mathbf{y}}_i) \right)
      \right\} - \frac{1}{n} \sum_{i=1}^n \mathbf{w}^\top \Psi(\mathbf{x}_i, \mathbf{y}_i)
      \right).
\end{equation}

Compared with the $n$-slack formulation described in Section~\ref{sec:nslackssvm}, 
the $1$-slack formulation of structured SVM increases the number of constraints exponentially~\cite{joachims2009cutting},
which means enumerating all constraints is also impractical.
Algorithm~\ref{alg:1slacktrain} described an approach similar to Algorithm~\ref{alg:nslacktrain} that uses a cutting-plane method to 
train the $1$-slack formulation of structured SVM.

\begin{algorithm}[htbp]
\caption{Cutting-plane algorithm for training $1$-slack formulation of structured SVM (with margin-rescaling)}
\label{alg:1slacktrain}
\begin{algorithmic}[1]
\STATE \textbf{Input}: $S = \left( (\mathbf{x}_1, \mathbf{y}_1), \dots, (\mathbf{x}_n, \mathbf{y}_n) \right),~ C,~ \varepsilon$
\STATE $\mathcal{W} = \emptyset$
%\REPEAT
\FOR{$k = 1,\dots,+\infty$}
    \STATE Generate query point $q^{(k)} = (\mathbf{w}^{(k)}, \xi^{(k)})$ by solving QP~(\ref{eq:1slackform}) w.r.t. all constraints in $\mathcal{W}$
    \STATE $\triangleright$ Query the oracle at point $q^{(k)}$ as follows
    \STATE Do loss-augmented inference:~
           $\hat{\mathbf{y}}_i^{(k)} = \argmax_{\bar{\mathbf{y}} \in \mathcal{Y}} \left\{ \Delta(\mathbf{y}_i, \bar{\mathbf{y}}) + 
            \langle \mathbf{w}^{(k)},~ \Psi(\mathbf{x}_i, \bar{\mathbf{y}}) \rangle \right\},~ \forall i$
    \IF{~$q^{(k)}$ is $\varepsilon$-feasible:~ $\frac{1}{n} \sum_{i=1}^n 
         \langle \mathbf{w}^{(k)},~ \Psi(\mathbf{x}_i, \mathbf{y}_i) - \Psi(\mathbf{x}_i, \hat{\mathbf{y}}_i^{(k)}) \rangle + \varepsilon \ge 
         \frac{1}{n} \sum_{i=1}^n \Delta(\mathbf{y}_i, \hat{\mathbf{y}}_i^{(k)}) - \xi^{(k)}$~}
        \RETURN $q^{(k)}$
    \ELSE
        \STATE Form a \emph{feasibility cut} and update constraints:~
               $\mathcal{W} = \mathcal{W} \cup \left\{ 
                \frac{1}{n} \sum_{i=1}^n \langle \mathbf{w},~ \Psi(\mathbf{x}_i, \mathbf{y}_i) - \Psi(\mathbf{x}_i, \hat{\mathbf{y}}_i^{(k)}) \rangle \ge 
                \frac{1}{n} \sum_{i=1}^n \Delta(\mathbf{y}_i, \hat{\mathbf{y}}_i^{(k)}) - \xi \right\}$
    \ENDIF
%\UNTIL{$\frac{1}{n} \sum_{i=1}^n 
%        \left( \mathbf{w}^\top \Psi(\mathbf{x}_i, \mathbf{y}_i) - \mathbf{w}^\top \Psi(\mathbf{x}_i, \hat{\mathbf{y}}_i) \right) + 
%        \varepsilon \ge \frac{1}{n} \sum_{i=1}^n \Delta(\mathbf{y}_i, \hat{\mathbf{y}}_i) - \xi$}
%\RETURN $(\mathbf{w}, \xi)$
\ENDFOR
\end{algorithmic}
\end{algorithm}


\section{Discussion}
\label{sec:ssvm_discussion}

From Algorithm~\ref{alg:nslacktrain} and Algorithm~\ref{alg:1slacktrain}, we observe that:
\begin{itemize}
\item To generate a query point $q$, it solves a QP with the same objective as the original optimisation problem and
      all constraints/cuts returned by previous queries. 
\item The Wolfe-dual programs of both QP (\ref{eq:nslackform}) and QP (\ref{eq:1slackform}) are QPs~\cite{tsochantaridis2005large,joachims2009cutting}.
\item All cutting-planes returned by the oracle are \emph{feasibility cuts}.
\item The training algorithm will \emph{stop} if the current query point $q$ is feasible, 
      in other words, it does not explicitly form an \emph{objective cut} when $q$ is feasible,
      which is reasonable as the algorithms optimise the objective when generating each query point.
\end{itemize}


\subsection{Query generation method}
\label{sec:ssvm_query}

Recall that in Section~\ref{sec:cuttingplane}, we have an objective $f_0(z)$ to minimise, in the case of $1$-slack formulation of structured SVM,
$f_0(z)$ is the quadratic objective in Equation~(\ref{eq:1slackform}), 
\begin{equation}
\label{eq:optobj}
f_0(z) = \frac{1}{2} \mathbf{w}^\top \mathbf{w} + C\xi,
\end{equation}
where $z = [\mathbf{w}, \xi]^\top$.
For the $n$-slack formulation, 
\begin{equation}
\begin{aligned}
f_0(z) = \frac{1}{2} \mathbf{w}^\top \mathbf{w} + \frac{C}{n} \sum_{i=1}^n \xi_i.
\end{aligned}
\end{equation}

The query point generation of both formulation of SSVM can be written as
\begin{equation*}
\begin{aligned}
\min_{z} ~& f_0(z) \\
s.t.~~ ~& A_k^\top z \le \mathbf{b}_k,
\end{aligned}
\end{equation*}
where $A_k^\top z \le \mathbf{b}_k$ is equivalent to the set of constraints in $\mathcal{W}$.


\subsection{Explicit objective cut generation}
\label{sec:ssvm_objcut}

Given query point $q = \left[ \mathbf{w}^{(k)}, \xi^{(k)} \right]^\top$, if $q$ is feasible, we can form an \emph{objective cut}
\begin{equation}
\label{eq:objcut_1slack}
\begin{aligned}
 & \nabla f_0(q)^\top (z - q) \\
=& \left[ \left[ \left.\frac{\partial f_0}{\partial \mathbf{w}}\right|_{\mathbf{w} = \mathbf{w}^{(k)}}, 
                 \left.\frac{\partial f_0}{\partial \xi}\right|_{\xi = \xi^{(k)}} \right]^\top \right]^\top 
   \left( \left[ \mathbf{w}, \xi \right]^\top - \left[ \mathbf{w}^{(k)}, \xi^{(k)} \right]^\top \right)  \\
=& \left[ \mathbf{w}^{(k)}, C \right] \left[ \mathbf{w} - \mathbf{w}^{(k)},~ \xi - \xi^{(k)} \right]^\top  \\
=& \langle \mathbf{w}^{(k)},~ \mathbf{w} - \mathbf{w}^{(k)} \rangle + C (\xi - \xi^{(k)}) \le 0.
\end{aligned}
\end{equation}

We have a similar \emph{objective cut} for the $n$-slack formulation of structured SVM
\begin{equation}
\label{eq:objcut_nslack}
\langle \mathbf{w}^{(k)}, \mathbf{w} - \mathbf{w}^{(k)} \rangle + \frac{C}{n} \sum_{i=1}^n (\xi_i - \xi_i^{(k)}) \le 0.
\end{equation}


\eat{
\subsubsection{Feasibility cut}
\label{sec:ssvm_feacut}

On the other hand, if $q = (\mathbf{w}^{(k)}, \xi^{(k)})$ is not feasible, the following constraint must be violated by $q$,
\begin{equation}
\label{eq:cut_1slackssvm}
\frac{1}{n} \sum_{i=1}^n \langle \mathbf{w},~ \Psi(\mathbf{x}_i, \mathbf{y}_i) - \Psi(\mathbf{x}_i, \hat{\mathbf{y}}_i^{(k)}) \rangle \ge 
\frac{1}{n} \sum_{i=1}^n \Delta(\mathbf{y}_i, \hat{\mathbf{y}}_i^{(k)}) - \xi.
\end{equation}
Let 
\begin{equation}
\label{eq:constraint_k}
f_k(z) = \frac{1}{n} \sum_{i=1}^n \Delta(\mathbf{y}_i, \hat{\mathbf{y}}_i^{(k)}) - 
         \frac{1}{n} \sum_{i=1}^n \langle \mathbf{w},~ \Psi(\mathbf{x}_i, \mathbf{y}_i) - \Psi(\mathbf{x}_i, \hat{\mathbf{y}}_i^{(k)}) \rangle - \xi,
\end{equation}
where $z = [\mathbf{w}, \xi]^\top$.
We can rewrite constraint (\ref{eq:cut_1slackssvm}) as $f_k(z) \le 0$.
Since $q$ violates this constraint, we can construct a \emph{feasibility cut}
\begin{equation}
\label{eq:feacut_1slack}
\begin{aligned}
 & f_k(q) + \nabla f_k(q)^\top (z - q) \\
=& f_k(q) + 
   \left[ \left[ \left.\frac{\partial f_k}{\partial \mathbf{w}}\right|_{\mathbf{w} = \mathbf{w}^{(k)}}, 
                 \left.\frac{\partial f_k}{\partial \xi}\right|_{\xi = \xi^{(k)}} \right]^\top \right]^\top 
   \left( \left[ \mathbf{w}, \xi \right]^\top - \left[ \mathbf{w}^{(k)}, \xi^{(k)} \right]^\top \right)  \\
=& f_k(q) + \left[ -\frac{1}{n} \sum_{i=1}^n \left( \Psi(\mathbf{x}_i, \mathbf{y}_i) - \Psi(\mathbf{x}_i, \hat{\mathbf{y}}_i^{(k)}) \right),  -1 \right] 
   \left[ \mathbf{w} - \mathbf{w}^{(k)},~ \xi - \xi^{(k)} \right]^\top  \\
=& \frac{1}{n} \sum_{i=1}^n \Delta(\mathbf{y}_i, \hat{\mathbf{y}}_i^{(k)}) - 
   \frac{1}{n} \sum_{i=1}^n \langle \mathbf{w}^{(k)},~ \Psi(\mathbf{x}_i, \mathbf{y}_i) - \Psi(\mathbf{x}_i, \hat{\mathbf{y}}_i^{(k)}) \rangle - 
   \xi^{(k)} + \langle -\frac{1}{n} \sum_{i=1}^n \left( \Psi(\mathbf{x}_i, \mathbf{y}_i) - \Psi(\mathbf{x}_i, \hat{\mathbf{y}}_i^{(k)}) \right),~
   \mathbf{w} - \mathbf{w}^{(k)} \rangle - \left( \xi - \xi^{(k)} \right)  \\
=& \frac{1}{n} \sum_{i=1}^n \Delta(\mathbf{y}_i, \hat{\mathbf{y}}_i^{(k)}) - 
   \frac{1}{n} \sum_{i=1}^n \langle \mathbf{w},~ \Psi(\mathbf{x}_i, \mathbf{y}_i) - \Psi(\mathbf{x}_i, \hat{\mathbf{y}}_i^{(k)}) \rangle - \xi \le 0.
\end{aligned}
\end{equation}

We found that inequalities (\ref{eq:cut_1slackssvm}) and (\ref{eq:feacut_1slack}) are identical.
This is \emph{not unexpected} as the hyperplane tangent to $f_k(z)$ (also a hyperplane) at point $q$ is \emph{identical} to hyperplane $f_k(z)$
(assuming the same domain for $z$). 

Similarly, for the $n$-slack formulation of structured SVM, 
suppose a constraint related to example $(\mathbf{x}_j, \mathbf{y}_j)$ is violated by query point $q$, 
as described in Algorithm~\ref{alg:nslacktrain}, the feasibility cut becomes
\begin{equation}
\label{eq:feacut_nslack}
g_k(z) = \Delta(\mathbf{y}_j, \hat{\mathbf{y}}^{(k)}) - 
\langle \mathbf{w},~ \Psi(\mathbf{x}_j, \mathbf{y}_j) - \Psi(\mathbf{x}_j, \hat{\mathbf{y}}^{(k)}) \rangle - \xi_j \le 0.
\end{equation}
}


\subsection{Generate query point using the method of Kelley-Cheney-Goldstein and the Chebyshev center method}
\label{sec:compare}

Suppose we use the method of Kelley-Cheney-Goldstein or the Chebyshev center method to generate query point 
when training the $1$-slack/$n$-slack formulation of structured SVM,
we can compare them with query point generation methods used in Algorithm~\ref{alg:nslacktrain} and Algorithm~\ref{alg:1slacktrain}.

Let 
\begin{align}
f_k(z) &= \frac{1}{n} \sum_{i=1}^n \Delta(\mathbf{y}_i, \hat{\mathbf{y}}_i^{(k)}) - 
          \frac{1}{n} \sum_{i=1}^n \langle \mathbf{w},~ \Psi(\mathbf{x}_i, \mathbf{y}_i) - \Psi(\mathbf{x}_i, \hat{\mathbf{y}}_i^{(k)}) \rangle - \xi
          \label{eq:constraint_k} \\
g_k(z) &= \Delta(\mathbf{y}_j, \hat{\mathbf{y}}^{(k)}) - 
          \langle \mathbf{w},~ \Psi(\mathbf{x}_j, \mathbf{y}_j) - \Psi(\mathbf{x}_j, \hat{\mathbf{y}}^{(k)}) \rangle - \xi_j \le 0 
          \label{eq:feacut_nslack}
\end{align}



\subsubsection{The $1$-slack formulation}
\label{sec:compare_1slack}

Given query points $q^{(1)}, \dots, q^{(k)}$ and the feasibility cuts returned by oracle (after querying these points), then
\begin{align*}
 & f_0(q^{(k)}) + \nabla f_0(q^{(k)})^\top (z - q^{(k)}) \\
=& \frac{1}{2} \langle \mathbf{w}^{(k)},~ \mathbf{w}^{(k)} \rangle + C\xi^{(k)} + 
   \left[ \mathbf{w}^{(k)}, C \right] \left[ \mathbf{w} - \mathbf{w}^{(k)},~ \xi - \xi^{(k)} \right]^\top  \\
=& \langle \mathbf{w}^{(k)}, \mathbf{w} \rangle - \frac{1}{2} \langle \mathbf{w}^{(k)}, \mathbf{w}^{(k)} \rangle + C\xi,
\end{align*}
where $q^{(k)} = (\mathbf{w}^{(k)}, \xi^{(k)})$ and $f_0(\cdot)$ is defined in Equation~(\ref{eq:optobj}).

If we use the method of \emph{Kelley-Cheney-Goldstein} (Section~\ref{sec:kcg}) to generate the next query point $q^{(k+1)}$,
we need to solve the following optimisation problem,
\begin{equation}
\label{eq:1slack_kcg}
\begin{aligned}
\min_{z} ~& \theta \\
s.t.~~ ~& \theta \ge \langle \mathbf{w}^{(k)}, \mathbf{w} \rangle - \frac{1}{2} \langle \mathbf{w}^{(k)}, \mathbf{w}^{(k)} \rangle + C\xi,~ \forall k \\
        & f_k(z) \le 0,~ \forall k \\
        & -\xi \le 0,
\end{aligned}
\end{equation}
where $z = [\mathbf{w}, \xi]^\top$ and $f_k(z)$ is defined in Equation~(\ref{eq:constraint_k}).

We need to solve a similar problem if we use the \emph{Chebyshev center} method (Section~\ref{sec:chebyshev}) to generate the next query point,
\begin{equation}
\label{eq:1slack_chebyshev}
\begin{aligned}
\min_{z} ~& \theta \\
s.t.~~ ~& \theta \ge 
          \frac{1}{D_k} \langle \mathbf{w}^{(k)}, \mathbf{w} \rangle + 
          (\frac{1}{2} - \frac{1}{D_k}) \langle \mathbf{w}^{(k)}, \mathbf{w}^{(k)} \rangle + 
          \frac{C}{D_k}\xi + C (1 - \frac{1}{D_k}) \xi^{(k)},~ \forall k \\
        & f_k(z) \le 0,~ \forall k \\
        & -\xi \le 0,
\end{aligned}
\end{equation}
where $D_k = \|\nabla f_0(q^{(k)})\| = \sqrt{\langle \mathbf{w}^{(k)}, \mathbf{w}^{(k)} \rangle + C^2}$ is a normalisation constant.

The method to generate the next query point used in Algorithm~\ref{alg:1slacktrain} can be rewritten as
\begin{equation}
\label{eq:1slack_query}
\begin{aligned}
\min_{z} ~& \frac{1}{2} \mathbf{w}^\top \mathbf{w} + C \xi \\
s.t.~~ ~& f_k(z) \le 0,~ \forall k \\
        & -\xi \le 0.
\end{aligned}
\end{equation}

Since $f_k(z)$ is a linear function, we know that both problem (\ref{eq:1slack_kcg}) and (\ref{eq:1slack_chebyshev}) are linear programs (LP),
and problem (\ref{eq:1slack_query}) is a quadratic program (QP). 
%We can see the difference clearly if we rewrite the objective of problem (\ref{eq:1slack_query}) to $\min_{z}\theta$ and add a new constraint
%$\theta \ge \frac{1}{2} \mathbf{w}^\top \mathbf{w} + C \xi$.
We can see the difference clearly from the \emph{epigraph form} of problem (\ref{eq:1slack_query}) 
\begin{align*}
\min_{z} ~& \theta \\
s.t.~~ ~& \theta \ge \frac{1}{2} \mathbf{w}^\top \mathbf{w} + C \xi \\
        & f_k(z) \le 0,~ \forall k \\
        & -\xi \le 0.
\end{align*}


\subsubsection{The $n$-slack formulation}
\label{sec:compare_nslack}

Similarly, for the $n$-slack formulation of structured SVM,
if we use the method of \emph{Kelley-Cheney-Goldstein} (Section~\ref{sec:kcg}) to generate the next query point $q^{(k+1)}$,
we need to solve an optimisation problem,
\begin{equation}
\label{eq:nslack_kcg}
\begin{aligned}
\min_{z} ~& \theta \\
s.t.~~ ~& \theta \ge \langle \mathbf{w}^{(k)}, \mathbf{w} \rangle - \frac{1}{2} \langle \mathbf{w}^{(k)}, \mathbf{w}^{(k)} \rangle + 
\frac{C}{n} \langle \mathbf{1}, \bm{\xi} \rangle,~ \forall k \\
        & g_k(z) \le 0,~ \forall k \\
        & -\xi_i \le 0,~ i = 1, \dots, n
\end{aligned}
\end{equation}
where $z = [\mathbf{w}, \bm{\xi}]^\top$, $\mathbf{1}$ is a $n$ dimensional vector of all $1$'s,
and $g_k(z)$ is defined in Equation~(\ref{eq:feacut_nslack}).


We need to solve a similar problem if we use the \emph{Chebyshev center} method (Section~\ref{sec:chebyshev}) to generate the next query point,
\begin{equation}
\label{eq:nslack_chebyshev}
\begin{aligned}
\min_{z} ~& \theta \\
s.t.~~ ~& \theta \ge 
          \frac{1}{G_k} \langle \mathbf{w}^{(k)}, \mathbf{w} \rangle + 
          (\frac{1}{2} - \frac{1}{G_k}) \langle \mathbf{w}^{(k)}, \mathbf{w}^{(k)} \rangle + 
          \frac{C}{nG_k} \langle \mathbf{1}, \bm{\xi} \rangle + 
          \frac{C}{n} (1 - \frac{1}{G_k}) \langle \mathbf{1}, \bm{\xi}^{(k)} \rangle,~ \forall k \\
        & g_k(z) \le 0,~ \forall k \\
        & -\xi_i \le 0,~ i = 1, \dots, n
\end{aligned}
\end{equation}
here 
$G_k 
= \sqrt{\langle \mathbf{w}^{(k)}, \mathbf{w}^{(k)} \rangle + \langle \frac{C}{n} \mathbf{1}, \frac{C}{n} \mathbf{1} \rangle} 
= \sqrt{\langle \mathbf{w}^{(k)}, \mathbf{w}^{(k)} \rangle + \frac{C^2}{n}}$
is a normalisation constant.

The method to generate the next query point used in Algorithm~\ref{alg:1slacktrain} can be rewritten (the epigraph form) as
\begin{equation}
\label{eq:nslack_genquery}
\begin{aligned}
\min_{z} ~& \theta \\
s.t.~~ ~& \theta \ge \frac{1}{2} \mathbf{w}^\top \mathbf{w} + \frac{C}{n} \langle \mathbf{1}, \bm{\xi} \rangle \\
        & g_k(z) \le 0,~ \forall k \\
        & -\xi_i \le 0.~ i = 1, \dots, n
\end{aligned}
\end{equation}

We observe similar differences as described in Section~\ref{sec:compare_1slack}.


\subsection{Efficient training via dualisation}
\label{sec:innerdual}

Recall that in Section~\ref{sec:nslackssvm}, 
the seperation oracle solves a loss-augmented inference problem for each query (Algorithm~\ref{alg:nslacktrain}),
which significantly reduces the scalability of the training algorithm.
Techniques have been developped to overcome this repeated inference, 
by exploring the dual problem of either the loss-augmented inference or 
the hinge-loss~\cite{taskar2004dissertation,taskar2005learning,meshi2010learning, bach2015paired},
which we review briefly in this section.

The $n$-slack formulation of structured SVM~(\ref{eq:nslackform}) described in Section~\ref{sec:nslackssvm} is equivalent to
\begin{align}
\min_{\mathbf{w}, ~\bm{\xi} \ge 0} ~& \frac{1}{2} \mathbf{w}^\top \mathbf{w} + \frac{C}{n} \sum_{i=1}^n \xi_i \\
s.t.~~ ~& \mathbf{w}^\top \Psi(\mathbf{x}_i, \mathbf{y}_i) + \xi_i \ge
          \max_{\bar{\mathbf{y}} \in \mathcal{Y}_i} 
          \left\{\mathbf{w}^\top \Psi(\mathbf{x}_i, \bar{\mathbf{y}}) + \Delta(\mathbf{y}_i, \bar{\mathbf{y}}) \right\},~\forall i. \label{eq:lossauginf}
\end{align}

If we can find a \emph{concise} formulation of the right-hand side of Equation~\ref{eq:lossauginf} (i.e., the loss-augmented inference),
in other words, the number of variables and constraints in the formulation is \emph{polynomial} in $L_i$, the number of variables in $\mathbf{y}_i$,
we can write its Lagrangian dual problem as 
\begin{align*}
\min_{\bm{\lambda}_i \ge \mathbf{0}} ~& h_i(\mathbf{w}, \bm{\lambda}_i) \\
s.t.~~ ~& g_i(\mathbf{w}, \bm{\lambda}_i) \le 0,
\end{align*}
where $h_i(\cdot)$ and $g_i(\cdot)$ are convex in both $\mathbf{w}$ and $\bm{\lambda}_i$.
Combining this minimisation over $\bm{\lambda}_i$ with the minimisation over $\mathbf{w}$ and $\bm{\xi}$,
we have a joint and compact convex minimisation problem
\begin{equation}
\label{eq:dualinf}
\begin{aligned}
\min_{\mathbf{w}, \bm{\xi}, \bm{\lambda}} ~& \frac{1}{2} \mathbf{w}^\top \mathbf{w} + \frac{C}{n} \sum_{i=1}^n \xi_i \\
s.t.~~ ~& \mathbf{w}^\top \Psi(\mathbf{x}_i, \mathbf{y}_i) + \xi_i \ge h_i(\mathbf{w}, \bm{\lambda}_i), ~\forall i \\
        & g_i(\mathbf{w}, \bm{\lambda}_i) \le 0, ~\forall i \\
        & \bm{\xi} \ge \mathbf{0}, ~\bm{\lambda}_i \ge \mathbf{0}, ~\forall i.
\end{aligned}
\end{equation}

Problem (\ref{eq:dualinf}) is a quadratic program with polynomial number of variables and constraints, 
and can be solved using existing off-the-shelf QP solvers.
Details of this approach are available in \cite{taskar2005learning}.
