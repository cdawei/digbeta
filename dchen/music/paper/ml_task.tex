\section{Machine learning tasks}

Suppose we have $N$ playlists from $U$ users, let $P_u$ be the set of indices of playlists from user $u$.
Let $M$ denote the total number of songs that was played at least once in any of the $N$ playlists.


\paragraph{Playlist augmentation}

We formulate the task of playlist augmentation as either a \emph{bipartite ranking} or \emph{binary classification} problem ({\it multitask?}),
that is, for each song that is not in the given playlist, 
we predict whether it will be added to the given playlist.
This formulation is illustrated in Figure~\ref{fig:pla},
where rows represent songs (no specific order) and columns represent playlists (no specific order).
Further, columns with white colour represent playlists in training set, 
and columns with grey colour represent playlists that should be augmented (\ie test set).
If entry $(i, j)$ is \texttt{1} (or \texttt{0}), 
it means the $i$-th song is (or not) found in the $j$-th playlist, 
and a question mark \texttt{?} means that we do not know whether the $i$-th song is found in the $j$-th playlist.
As a remark, columns represent playlists in test set contain only \texttt{1} and \texttt{?} entries.

\begin{figure}[!h]
\centering
\newcolumntype{a}{>{\columncolor[gray]{0.9}}c}  % define a new column type
\setlength{\tabcolsep}{1pt} % tweak the space between columns
%\begin{tabular}{|*{7}{c}|ccccc|} \hline
\begin{tabular}{|*{7}{c}|aaaaa|} \hline
%\rule{.3em}{0pt} 
\rule{0em}{10pt}
& \texttt{0} & \texttt{1} & \texttt{0} & $\cdots$ & \texttt{0} & & & \texttt{?} & $\cdots$ & \texttt{?} & \\
& \texttt{0} & \texttt{0} & \texttt{0} & $\cdots$ & \texttt{0} & & & \texttt{1} & $\cdots$ & \texttt{?} & \\
& \texttt{1} & \texttt{0} & \texttt{0} & $\cdots$ & \texttt{0} & & & \texttt{?} & $\cdots$ & \texttt{?} & \\
%\vspace{-5pt}
\vspace{-3pt}
& \texttt{0} & \texttt{0} & \texttt{0} & $\cdots$ & \texttt{1} & & & \texttt{?} & $\cdots$ & \texttt{1} & \\
& $\vdots$ & $\vdots$ & $\vdots$ & $\vdots$ & $\vdots$ & & & $\vdots$ & $\vdots$ & $\vdots$ & \\
& \texttt{0} & \texttt{0} & \texttt{1} & $\cdots$ & \texttt{0} & & & \texttt{?} & $\cdots$ & \texttt{?} & \\ \hline
\end{tabular}
\caption{Illustration of playlist augmentation as multi-label classification.}
\label{fig:pla}
\end{figure}



\paragraph{New song recommendation}

We formulate the task of new song recommendation as either a \emph{bipartite ranking} or \emph{binary classification} problem ({\it multitask?}),
where we predict, for each song in test set,
whether it will be included in a given playlist.
This formulation is illustrated in Figure~\ref{fig:mlr},
where rows represent songs (from top to bottom, sorted by the release date in ascending order)
and columns represent playlists (no specific order).
Further, rows with white colour represent songs in training set, and rows with grey colour represent songs in test set.
If entry $(i, j)$ is \texttt{1} (or \texttt{0}), it means the $i$-th song is (or not) found in the $j$-th playlist,
otherwise, we do not know whether the $i$-th song is found in the $j$-th playlist (\ie entry $(i, j)$ is a question mark \texttt{?}).
As a remark, we do not care about the order of songs in a playlist.

\begin{figure}[!h]
\centering
\setlength{\tabcolsep}{1pt} % tweak the space between columns
\begin{tabular}{|*{10}{c}|} \hline
%\rule{.3em}{0pt} 
\rule{0em}{10pt}
& \texttt{0} & \texttt{0} & \texttt{0} & \texttt{1} & \texttt{0} & $\cdots$ & $\cdots$ & \texttt{0} & \\
& \texttt{1} & \texttt{0} & \texttt{0} & \texttt{0} & \texttt{0} & $\cdots$ & $\cdots$ & \texttt{0} & \\
\vspace{-5pt}
& \texttt{0} & \texttt{0} & \texttt{0} & \texttt{0} & \texttt{0} & $\cdots$ & $\cdots$ & \texttt{1} & \\
& $\vdots$ & $\vdots$ & $\vdots$ & $\vdots$ & $\vdots$ & $\vdots$ & $\vdots$ & $\vdots$ & \\
& \texttt{0} & \texttt{0} & \texttt{0} & \texttt{0} & \texttt{1} & $\cdots$ & $\cdots$ & \texttt{0} & \\ \hline
%\rowcolor{gray!20}
\rowcolor[gray]{0.9}
%\vspace{-5pt}
\vspace{-3pt}
& \texttt{?} & \texttt{?} & \texttt{?} & \texttt{?} & \texttt{?} & $\cdots$ & $\cdots$ & \texttt{?} & \rule{0em}{10pt} \\
%\rowcolor{gray!20}
\rowcolor[gray]{0.9}
& $\vdots$ & $\vdots$ & $\vdots$ & $\vdots$ & $\vdots$ & $\vdots$ & $\vdots$ & $\vdots$ &  \\
%\rowcolor{gray!20}
\rowcolor[gray]{0.9}
& \texttt{?} & \texttt{?} & \texttt{?} & \texttt{?} & \texttt{?} & $\cdots$ & $\cdots$ & \texttt{?} & \\ \hline
\end{tabular}
\caption{Illustration of recommending new songs to augment playlists as multi-label classification.}
\label{fig:mlr}
\end{figure}



\paragraph{Song scoring}

The score of song $m$ in playlist $i$ from user $u$ is a linear form,
\begin{equation*}
\begin{aligned}
f(u, i, m) &= \bar\w_{u,i}^\top \x_m,
\end{aligned}
\end{equation*}
where $\x_m$ is a feature vector of song $m$ which contains audio features, song genre, and 
\begin{equation*}
\bar\w_{u,i} = \bv_u + \w_i + \mubm,
\end{equation*}
where $\bv_u$ is a weight vector specific to user $u$,
$\w_i$ is a weight vector specific to playlist $i$,
and $\mubm$ is a weight vector shared by all users.

Further, when recommending songs for user $u$, we can also utilise song popularity as well as the listening events of all other users as features,
if these are available at test time (\eg playlist augmentation), \ie
\begin{equation*}
\begin{aligned}
f(u, i, m) &= \bar\w_{u,i}^\top \phibm_{u,m},
\end{aligned}
\end{equation*}
where $\phibm_{u,m}$ is a new feature vector concatenating both $\x_m$ and the number of listening events by all users except $u$.
