\section{Related work}
% playlist recommendation
% only describe problem settings: playlist generation, next song recommendation, playlist continuation

We summarise recent work most related to playlist recommendation and music recommendation in cold-start 
scenarios,
as well as work on the connection between bipartite ranking and binary classification.

%\subsubsection{Playlist recommendation}
There is a rich collection of recent literature on playlist recommendation,
%\cite{platt2002learning,mcfee2011natural,mcfee2012hypergraph,chen2012playlist,ben2017groove,
%hariri2012context,bonnin2013evaluating,jannach2015beyond,
%schedl2017,recsysch2018},
which can be summarised in two typical settings: 
playlist generation and next song recommendation. % and playlist continuation.
Playlist generation is to generate a complete playlist given some 
seed. %~\cite{platt2002learning,mcfee2011natural,mcfee2012hypergraph,chen2012playlist,ben2017groove},
For example, the AutoDJ system~\cite{platt2002learning} generates playlists given one or more seed songs,
Groove Radio can produce a personalised playlist for the specified user given a seed artist~\cite{ben2017groove},
%a playlist for a specific user can be generated by Groove Radio given a seed artist~\cite{ben2017groove},
or a seed location in hidden space, where all songs are embedded,
can be specified in order to generate a complete playlist~\cite{chen2012playlist}.
There are also work that focus on evaluating the learned playlist model,
without concretely generating playlists~\cite{mcfee2011natural,mcfee2012hypergraph}.
See this recent survey~\cite{bonnin2015automated} for more details.
%
Next song recommendation %~\cite{hariri2012context,bonnin2013evaluating,jannach2015beyond}
is to predict the next song a user might play after observing some context.
For example, the most recent sequence of songs with which a user has interacted was used to
%the most recent sequence of songs a user interacted with the system was used to
infer the contextual information. %which 
This was further adopted to rank the next possible song
with regards to a topic-based sequential patterns learned from users' playlists~\cite{hariri2012context}.
Context can also be the artists appeared in user's listening history,
%The artists appeared in user's listening history can be used as context,
which, together with the popularity of songs or frequency of artists collocations,
were employed to score the next song~\cite{mcfee2012million,bonnin2013evaluating}.
It is straightforward to produce a complete playlist using next song recommendation techniques,
\ie by picking the next song sequentially~\cite{bonnin2013evaluating,ben2017groove}.
%It is obvious that next song recommendation techniques can also be used to generate a
%complete playlist by picking the next song sequentially~\cite{bonnin2013evaluating,ben2017groove}.
%
%Playlist continuation is to add one or more songs to a playlist,
%given the added songs serve the same purpose of the original playlist~\cite{schedl2017}. %,recsysch2018}.
%One may notice that playlist generation and next song recommendation are two special cases
%of playlist continuation, which means techniques developed for playlist generation and
%next song recommendation can also be used for playlist continuation.


% cold start recommendations
% copy the cold start paragraph in related work
% \subsubsection{Cold-start recommendation}
In the collaborative filtering literature,
the cold-start setting has primarily been addressed through
suitable regularisation of matrix factorisation parameters
based on exogenous user- or item-features~\cite{Ma:2008,Agarwal:2009,Cao:2010}.
%Another popular approach involves explicitly mapping such features to the latent embeddings~\cite{Gantner:2010}.
%
Content-based approaches~\cite[chap. 4]{aggarwal2016recommender}
can handle the recommendation of new songs,
%can be adopted to recommend {\it cold songs} (\ie new songs),
typically by making use of content features of songs extracted either automatically~\cite{seyerlehner2010automatic,eghbal2015vectors}
or manually by musical experts~\cite{john2006pandora}.
Further, content features can also be combined with other approaches, such as those based on 
collaborative filtering~\cite{yoshii2006hybrid,donaldson2007hybrid,shao2009music},
which is known as the hybrid recommendation approaches~\cite{burke2002hybrid,aggarwal2016recommender}.
%see~\cite[Chapter~6]{aggarwal2016recommender} for a general description. % besides music recommendation.
%
%The problem of recommending music for %{\it cold users} (\ie new users) 
%new users has been tackled by a number of approaches, including transferring user preferences learned 
%from related domains~\cite{hu2010study,aizenberg2012build},
%and techniques that balance the exploration-exploitation trade-off~\cite{wang2014exploration,liebman2015dj}.

Another popular approach for cold-start recommendation involves explicitly mapping 
user- or item- content features to latent embeddings~\cite{Gantner:2010}.
This approach can be adopted to recommend new songs, 
\eg by learning a convolutional neural network to map audio features of new songs to 
the corresponding latent embeddings~\cite{van2013deep},
% has been adopted in music recommendation~\cite{van2013deep}
which are further used to score songs with the latent embeddings of playlists (learned by matrix factorisation).
The problem of recommending music for new users can also be tackled using a similar approach, \eg
%Similar approaches can be employed to make recommendations for new users 
by learning a mapping from user attributes to user embeddings.

A slightly different approach to deal with music recommendation for new users is learning hierarchical 
representations for genre, sub-genre and artist.
By adopting an additive form with user and artist weights, it can fall back to use only artist weights
when recommending music to new users, if artist weights are unavailable, it further falls back to use 
sub-genre (or genre) weights~\cite{ben2017groove}.
However, the requirement of seed information (\eg artist, genre or a seed song) restricts its direct applicability to
the \emph{cold playlists} and \emph{cold users} settings, %where no seed information is available.
further, encoding song usage information as features makes it unsuitable 
%to straight recommend new songs.
for recommending new songs directly.
%Further, as song usage information has been encoded as song features, this makes it unsuitable for recommending
%new songs where usage information is not available.
%
% keep this?
%There are other approaches to make recommendations for new users, such as transferring user preferences learned 
%from related domains~\cite{hu2010study,aizenberg2012build}, or balancing the exploration-exploitation 
%trade-off~\cite{wang2014exploration,liebman2015dj}.
%
%Other approaches to make recommendations for new users including transferring user preferences learned 
%from related domains~\cite{hu2010study,aizenberg2012build},
%and techniques that balance the exploration-exploitation trade-off~\cite{wang2014exploration,liebman2015dj}.



%It has been known 
It is well known
that bipartite ranking and binary classification are
closely related~\cite{ertekin2011equivalence,menon2016bipartite}.
In particular, \citet{ertekin2011equivalence} have shown that the P-Norm Push bipartite ranking loss~\cite{rudin2009p}
is equivalent to the P-Classification loss~\cite{ertekin2011equivalence} when using the exponential surrogate.
Further, the P-Norm Push loss is an approximation of the Infinite-Push loss~\cite{agarwal2011infinite},
or equivalently, the Top-Push loss~\cite{li2014top}, which focuses on the highest ranked negative example instead of
the lowest ranked positive example as in the Bottom-Push loss~\cite{rudin2009p} that we adopt in this work. %~(\ref{eq:bprisk}).
%
Compared to the Bayesian Personalised Ranking (BPR) approach which requires all
positive items to be ranked higher than those unobserved ones~\cite{rendle2009bpr,mcfee2012million}, 
the adopted approach penalises unobserved items that ranked higher than the lowest ranked positive item,
which can be optimised more efficiently when one cares about only the top ranked items~\cite{rudin2009p,li2014top}.
%a similar approach has been shown to work more efficiently when one cares about only the top ranked items~\cite{li2014top}.


% combined: cold start playlist recommendation
% We investigate three settings of cold-start problems in playlist recommendation
%The work that most related to ours is \citep{ben2017groove}, which learns hierarchical representations
%for genre, sub-genre and artist.
%It uses a similar additive form that composite of user weights and artist weights.
%To make cold start recommendation, it falls back to use only artist weights,
%if artist weights are unavailable, it further falls back to use sub-genre (or genre) weights.
%%However, this work requires additional context information when make cold start recommendations,
%This approach encodes song usage information as features which restricts its applicability for other settings
%of cold start settings considered in this work.

%Another work that can deal with cold start recommendation is \citep{van2013deep}.
%This approach can make cold start recommendation by mapping song audio features to song latent factors 
%with a convolutional neural network, which is further used to score new (cold) songs with 
%latent factors of playlists learned by matrix factorisation.
%Similar approaches has also been proposed in~\cite{Gantner:2010} which learns functions to map
%user/item attributes to the corresponding latent factors that learned by matrix factorisation.

% Groove radio paper: deal with cold user playlist recommendation given context
% Deep content music recommendation: deal with cold song/user recommendation if there's content attributes for song/user
% Similar to a general setting in ICDM'10 paper
