% playlist recommendation
% only describe problem settings: playlist generation, next song recommendation, playlist continuation
{\bf Playlist recommendation}.
There is rich collection of recent literature on the playlist 
recommendation,
%\cite{platt2002learning,mcfee2011natural,mcfee2012hypergraph,chen2012playlist,ben2017groove,
%hariri2012context,bonnin2013evaluating,jannach2015beyond,
%schedl2017,recsysch2018},
which can be summarised in three typical settings: 
playlist generation, next song recommendation, and playlist continuation.
Playlist generation is to generate a complete playlist given some seed~\cite{platt2002learning,mcfee2011natural,
mcfee2012hypergraph,chen2012playlist,ben2017groove},
for example, the AutoDJ system~\cite{platt2002learning} generates playlists given one or more seed songs,
a playlist for a specific user can be generated by Groove Radio given a seed artist~\cite{ben2017groove},
or a seed location in hidden space, where all songs are embedded,
can be specified in order to generate a complete playlist~\cite{chen2012playlist}.
There are also work that focus on evaluating the learned playlist model,
without concretely generate playlists~\cite{mcfee2011natural,mcfee2012hypergraph}.
See this recent survey~\cite{bonnin2015automated} for more details.

Next song recommendation~\cite{hariri2012context,bonnin2013evaluating,jannach2015beyond}
is to predict the next song a user might play after observing some context,
for example, the most recent sequence of songs a user interacted with the system was used to
infer the contextual information, which was further adopted to rank the next possible song
with regards to a topic-based sequential patterns learned from user playlists~\cite{hariri2012context}.
The artists appeared in user's listening history can be used as context,
which, together with the popularity of songs or frequency of artists collocations,
were employed to score the next song~\cite{mcfee2012million,bonnin2013evaluating}.
It is obvious that next song recommendation techniques can also be used to generate a
complete playlist by picking the next song sequentially~\cite{bonnin2013evaluating,ben2017groove}.

Playlist continuation is to add one or more songs to a playlist,
given the added songs serve the same purpose of the original playlist~\cite{schedl2017,recsysch2018}.
One may notice that playlist generation and next song recommendation are two special cases
of playlist continuation, which means techniques developed for playlist generation and
next song recommendation can also be used for playlist continuation.


% cold start recommendations
% copy the cold start paragraph in related work
{\bf Cold-start recommendation}.
In the collaborative filtering literature,
the cold-start setting has primarily been addressed through
suitable regularisation of matrix factorisation parameters
based on exogenous user- or item-features~\cite{Ma:2008,Agarwal:2009,Cao:2010}.
Another popular approach involves explicitly mapping such features to the latent embeddings~\cite{Gantner:2010}.

Content based recommendation approaches~\cite[Chapter~4]{aggarwal2016recommender}
can be adopted to recommend {\it cold songs} (\ie new songs),
typically by making use of content features of songs extracted either automatically~\cite{seyerlehner2010automatic,eghbal2015vectors}
or manually by musical experts~\cite{john2006pandora}.
Further, content features can also be combined with other approaches, such as those based on 
collaborative filtering~\cite{yoshii2006hybrid,donaldson2007hybrid,shao2009music}.
This is known as the hybrid recommendation approaches~\cite{burke2002hybrid}, 
see~\cite[Chapter~6]{aggarwal2016recommender} for a general description. % besides music recommendation.
The problem of recommending music for {\it cold users} (\ie new users) 
has been tackled by a number of approaches, including transferring user preferences learned 
from related domains~\cite{hu2010study,aizenberg2012build},
and techniques that balance the exploration-exploitation trade-off~\cite{wang2014exploration,liebman2015dj}.


% combined: cold start playlist recommendation
% We investigate three settings of cold-start problems in playlist recommendation
The work that most related to ours is \citep{ben2017groove}, which learns hierarchical representations
for genre, sub-genre and artist.
It uses a similar additive form that composite of user weights and artist weights.
To make cold start recommendation, it falls back to use only artist weights,
if artist weights are unavailable, it further falls back to use sub-genre (or genre) weights.
%However, this work requires additional context information when make cold start recommendations,
This approach encodes song usage information as features which restricts its applicability for other settings
of cold start settings considered in this work.

Another work that can deal with cold start recommendation is \citep{van2013deep}.
This approach can make cold start recommendation by mapping song audio features to song latent factors 
with a convolutional neural network, which is further used to score new (cold) songs with 
latent factors of playlists learned by matrix factorisation.
Similar approaches has also been proposed in~\cite{gantner2010learning} which learns functions to map
user/item attributes to the corresponding latent factors that learned by matrix factorisation.


% Groove radio paper: deal with cold user playlist recommendation given context
% Deep content music recommendation: deal with cold song/user recommendation if there's content attributes for song/user
% Similar to a general setting in ICDM'10 paper


