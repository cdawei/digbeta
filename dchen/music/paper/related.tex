%\section{Related work}
%\label{sec:related}

%\cheng{Make this a subsection of the multitask learning section.}

We first summarise recent work of recommending music to form playlists
according the problem settings, the recommendation methods,
as well as the information being utilised,
and then music recommendation techniques in cold-start scenarios.

%\cheng{Reduce related work subsection to 1 column}

{\bf Problem settings}.
There are three typical settings:
playlist generation, next song recommendation, and playlist continuation.
%
There is rich collection of recent literature on the playlist generation or prediction
problem~\cite{platt2002learning,mcfee2011natural,mcfee2012hypergraph,chen2012playlist,ben2017groove}.
%
Typical work is to generate a complete playlist given some seed,
for example, the AutoDJ system~\cite{platt2002learning} generates playlists given one or more seed songs,
a playlist for a specific user can be generated by Groove Radio given a seed artist~\cite{ben2017groove},
or a seed location in hidden space, where all songs are embedded,
should be specified in order to generate a complete playlist~\cite{chen2012playlist}.
%
There are also work that focus on evaluating the learned playlist model,
without concretely generate playlist~\cite{mcfee2011natural,mcfee2012hypergraph}.
More details of playlist generation can be found in this recent survey~\cite{bonnin2015automated}.


Next song recommendation~\cite{hariri2012context,bonnin2013evaluating,jannach2015beyond}
is to predict the next song a user might play after observing some context,
for example, the most recent sequence of songs a user interacted with the system was used to
infer the contextual information, which was further used to rank the next possible song
with regards to a topic-based sequential patterns learned from user playlists~\cite{hariri2012context}.
%
The artists appeared in user's listening history can be used as context,
which, together with the popularity of song or frequency of artists collocations,
were further used to score the next song~\cite{mcfee2012million,bonnin2013evaluating}.
%
It is obvious that next song recommendation techniques can also be used to generate a
complete playlist by picking the next song sequentially~\cite{bonnin2013evaluating,ben2017groove}.


Playlist continuation is to add one or more songs to a playlist,
given the added songs serve the same purpose of the original playlist~\cite{schedl2017,recsysch2018}.
One can immediately notice that playlist generation and next song recommendation are special cases
of playlist continuation, which means techniques developed for playlist generation and
next song recommendation can both be used for playlist continuation.

In the collaborative filtering literature,
the cold-start setting has primarily been addressed through
suitable regularisation of matrix factorisation parameters
based on exogenous user- or item-features~\cite{Ma:2008,Agarwal:2009,Cao:2010}.
Another popular approach involves explicitly mapping such features to the latent embeddings~\cite{Gantner:2010}.

{\bf Methods}.
Ranking approaches such as popularity-based ranking have been shown to
work surprisingly well~\cite{mcfee2012million,bonnin2013evaluating,bonnin2015automated}.
The reason is believed to be the long-tail distribution of songs in
playlists~\cite{cremonesi2010performance,bonnin2013evaluating}.
%
This approach has been further improved by taking into account artist information in addition to
song popularity, which creates a comparably strong baseline that outperforms many sophisticated
approaches such as Bayesian Personalised Ranking based approach, neighbourhood methods, and approaches
making use of association rules and sequential patterns~\cite{mcfee2012million,bonnin2013evaluating}.
%
Ranking method has also been used as a post-processing component in more sophisticated approaches,
where a subset of songs were selected by scoring~\cite{jannach2015beyond} or matching~\cite{hariri2012context}
before ranking which optimises specific characteristics of the generated playlists.


Markov chains and related approaches were also widely used for playlist generation by casting the task
as language modelling problem~\cite{mcfee2011natural},
random walks in a hyper-graph~\cite{mcfee2012hypergraph} where songs were first grouped (by genre,
year of release, or social tags etc.) to serve as edges in the hyper-graph, or samples of Markov chains
in latent space where songs are embedded as (pairs of) points~\cite{chen2012playlist}.
%
Other techniques including playlist generation as sequential classification based on context~\cite{ben2017groove},
Gaussian process regression for user preference learning~\cite{platt2002learning},
topic models for sequential pattern mining and the well-known matrix factorisation techniques for learning
latent representations of songs, artists and users~\cite{mcfee2012hypergraph,chen2012playlist,ben2017groove}.


{\bf A diverse set of information} has been used for music recommendation,
such as song metadata (\eg song title, artist name, era, genre, albums etc.)~\cite{hariri2012context,platt2002learning},
content data (\eg lyrics, low level audio features etc.)~\cite{mcfee2011natural,mcfee2012hypergraph,jannach2015beyond,ben2017groove},
artists information (\eg artist popularity, collocation of artists etc.)~\cite{bonnin2013evaluating,ben2017groove},
user listening history and social interactions (\eg social tags) as well as usage statistics (\eg song popularity,
song co-occurrence etc.)~\cite{mcfee2012hypergraph,hariri2012context,bonnin2013evaluating,jannach2015beyond,ben2017groove}.
There are a few work that make use of latent representations of song, user and artist which are learned from existing playlists,
or user-song and user-artist interactions~\cite{chen2012playlist,ben2017groove}.

The sequential order of songs in playlist has not been well understood~\cite{schedl2017},
some work suggest that song order and song-to-song transitions are important
for playlist quality~\cite{mcfee2012hypergraph,kamehkhosh2018automated},
while other work have shown that the ensemble of songs in playlist do matter,
but the song order seems to be negligible~\cite{tintarev2017sequences,vall2017importance}.
In this work, we treat a playlist as a set of songs by discarding the sequential order,
and leave the investigation of using song order to assist playlist generation as future work.

%\cheng{In 2-3 sentences, re-define the problem we are attacking in this paper.}



%It has been known that binary classification and bipartite ranking are
%closely related~\cite{ertekin2011equivalence,menon2016bipartite}.
%In particular, \citet{ertekin2011equivalence} have shown that the P-Norm Push bipartite ranking loss~\cite{rudin2009p}
%is equivalent to the P-Classification loss~\cite{ertekin2011equivalence} when using the exponential surrogate.
%Further, the P-Norm Push loss is an approximation of the Infinite-Push loss~\cite{agarwal2011infinite},
%or equivalently, the Top-Push loss~\cite{li2014top}, which focuses on the highest ranked negative example instead of
%of the lowest ranked positive example as in~(\ref{eq:bprisk}).
%
%Inspired by these connections, we seek a classification loss that is equivalent to a bipartite ranking loss (under a few assumptions),
%which can approximate the risk with Bottom-Push loss in~(\ref{eq:bprisk}).
%This will make it possible to formulate an unconstrained objective that approximates the empirical loss $\RCal_\textsc{rank}$.


{\bf Cold-start music recommendation}.
Content based recommendation approaches~\cite[Chapter~4]{aggarwal2016recommender}
can be adopted to recommend {\it cold songs} (\ie new songs),
typically by making use of content features of songs extracted either automatically~\cite{seyerlehner2010automatic,eghbal2015vectors}
or manually by musical experts~\cite{john2006pandora}.
Further, content features can also be combined with other approaches, such as those based on 
collaborative filtering~\cite{yoshii2006hybrid,donaldson2007hybrid,shao2009music}.
This is known as the hybrid recommendation approaches~\cite{burke2002hybrid}, 
see~\cite[Chapter~6]{aggarwal2016recommender} for a general description besides music recommendation.
The problem of recommending music for {\it cold users} (\ie new users) 
has also been tackled by a number of approaches, such as transferring user preferences learned 
from related domains~\cite{hu2010study,aizenberg2012build},
or techniques that balance the exploration-exploitation trade-off~\cite{wang2014exploration,liebman2015dj}.
