\section{Multitask learning for music recommendation}
\label{sec:method}

\cheng{Move this to paragraph, and figure, to Section 2. You need to discuss the problem we are solving before you can talk about related work.}

Suppose we have a dataset $\DCal$ with $N$ playlists from $U$ users, songs in every playlist are from a music collection
with $M$ songs, and each song in the collection is appeared in at least one playlist.

The task of recommending a set of newly released songs to extend an existing playlist for a given user is illustrated in Figure~\ref{fig:pla},
where rows represent songs (no specific order) and columns represent playlists (no specific order).
Further, columns with white colour represent playlists in training set,
and columns with grey colour represent playlists in test set.
If entry $(m, n)$ is \texttt{1} (or \texttt{0}),
it means song $m$ can (or not) be found in playlist $n$,
and a question mark \texttt{?} means we do not know whether song $m$ is in playlist $n$.
%As a remark, columns represent playlists in test set contain only \texttt{?} entries.

\begin{figure}[!h]
\centering
\setlength{\tabcolsep}{1pt} % tweak the space between columns
\begin{tabular}{|*{10}{c}|} \hline
%\rule{.3em}{0pt} 
\rule{0em}{10pt}
& \texttt{0} & \texttt{0} & \texttt{0} & \texttt{1} & \texttt{0} & $\cdots$ & $\cdots$ & \texttt{0} & \\
& \texttt{1} & \texttt{0} & \texttt{0} & \texttt{0} & \texttt{0} & $\cdots$ & $\cdots$ & \texttt{0} & \\
\vspace{-5pt}
& \texttt{0} & \texttt{0} & \texttt{0} & \texttt{0} & \texttt{0} & $\cdots$ & $\cdots$ & \texttt{1} & \\
& $\vdots$ & $\vdots$ & $\vdots$ & $\vdots$ & $\vdots$ & $\vdots$ & $\vdots$ & $\vdots$ & \\
& \texttt{0} & \texttt{0} & \texttt{0} & \texttt{0} & \texttt{1} & $\cdots$ & $\cdots$ & \texttt{0} & \\ \hline
%\rowcolor{gray!20}
\rowcolor[gray]{0.9}
%\vspace{-5pt}
\vspace{-3pt}
& \texttt{?} & \texttt{?} & \texttt{?} & \texttt{?} & \texttt{?} & $\cdots$ & $\cdots$ & \texttt{?} & \rule{0em}{10pt} \\
%\rowcolor{gray!20}
\rowcolor[gray]{0.9}
& $\vdots$ & $\vdots$ & $\vdots$ & $\vdots$ & $\vdots$ & $\vdots$ & $\vdots$ & $\vdots$ &  \\
%\rowcolor{gray!20}
\rowcolor[gray]{0.9}
& \texttt{?} & \texttt{?} & \texttt{?} & \texttt{?} & \texttt{?} & $\cdots$ & $\cdots$ & \texttt{?} & \\ \hline
\end{tabular}
\caption{Illustration of recommending new songs to augment playlists as multi-label classification.}
\label{fig:mlr}
\end{figure}


\cheng{Update Figure 1 to include meaning of rows and columns. Why do we bother with ?, since the whole test set is unknown? Indicate the effect of known and unknown users.}

\cheng{In 2-3 sentences, define the problem we are attacking in this paper. Recommend a set, not a single item. User specific. All songs known. Consider cold start user.}

\cheng{This should be followed by related work, where you need to describe for each group how we are different, or how we use some idea.}

\subsection{Multitask learning regularisation}

%\paragraph{Song scoring}
%
We aim to learn a function $f(m, n)$ that scores the affinity between song $m$ and playlist $n$.
Suppose function $f(m, n)$ has a linear form
\begin{equation}
\label{eq:scorefunc}
f(m, n) = \w_n^\top \x_m, \  m \in \{1,\dots,M\}, \, n \in \{1,\dots,N\},
\end{equation}
where vector $\w_n$ represents parameters of playlist $n$ and $\x_m$ is a feature vector of song $m$.

The learning task is to minimise the empirical risk of scoring function $f$ on dataset $\DCal$ over parameters $\W$.
Formally, we solve an optimisation problem
\begin{equation}
\label{eq:obj}
\min_{\W} \, \lambda_1 \sum_{n=1}^N \| w_n \|_2^2 + \RCal(f, \DCal),
\end{equation}
where $\lambda_1 > 0$ is the regularisation constant and $\RCal(f, \DCal)$ denotes the empirical risk.

We assume that playlists of the \emph{same} user have \emph{similar} representations,
and the difference between playlists $n$ and $n'$ of the same user $u$ can be reflected by the difference 
between their parameters, we thus introduce the following regularisation term 
\begin{equation}
\label{eq:mtreg}
\sum_{u=1}^U \sum_{(n, n') \in P_u} \| \w_n - \w_{n'} \|_2^2,
\end{equation}
where $P_u$ is the set of playlist pairs from user $u$.
We call (\ref{eq:mtreg}) a multitask regulariser, since it regularises the parameters of multiple playlists.

In summary, the regularisation term in our learning objective is
\begin{equation}
\label{eq:reg}
R(\W) = \lambda_1 \sum_{n=1}^N \| \w_n \|_2^2 + \lambda_2 \sum_{u=1}^U \sum_{(n, n') \in P_u} \| \w_n - \w_{n'} \|_2^2,
\end{equation}
where $\lambda_1, \lambda_2 \in \R_+$ are regularisation constants.

Once all parameters have been learned, we can make recommendation by first scoring each song in music collection
for a given playlist by our scoring function $f$, 
then we can extend the given playlist by either taking the top-K scored songs or sampling songs proportional to their scores.
