% !TEX root=./main.tex

\section{Introduction}
\label{sec:intro}
Online music streaming services (e.g., Spotify, Pandora, Apple Music) % Google Play Music, Amazon Music) 
are playing an increasingly important role in the digital music industry.
A key ingredient of these services is the ability to automatically recommend songs to help users explore large collections of music.
%as well as create an uninterrupted listening experience.
Such recommendation is often in the form of a \emph{playlist}, 
%which is simply an ordered sequence of songs
which involves a (small) set of songs.
%
We investigate the problem of recommending songs to form personalised playlists 
in \emph{cold-start} scenarios, %settings,
where there is no historical data for either users or songs.
%
% one paragraph about recommender system
Conventional recommender systems for books or movies~\citep{Sarwar:2001,Netflix}
typically learn a score function via matrix factorisation~\citep{Koren:2009},
and recommend the item that achieves the highest score.
This approach is not suited to %deal with 
cold-start settings
%\emph{cold-start} settings,
%where there is no historical data for either users or items
due to the lack of interaction data. %for either users or items.
%between users and items.
%
% explain what is different about recommending playlist
Further, in playlist recommendation,
one has to recommend a subset of a large collection of songs instead of only one top ranked song.
Enumerating all possible such subsets is intractable;
additionally,
it is likely that more than one playlist is satisfactory, since
users generally maintain more than one playlist when using a music streaming service,
which leads to challenges in standard supervised learning.


%We investigate the problem of recommending a playlist of songs for a given user.
%If the user is new to the system, we would like to make a decent default recommendation by learning from 
%all available playlists of existing users.
%On the other hand, if the user already has a few playlists in the system, it is expected the recommender 
%system can also learn from this information and hopefully make better recommendations.

%We investigate the problem of recommending songs to form personalised playlists %(for a given user)
%in cold-start settings by learning from user-curated playlists.

%First, we study the problem of recommending a set of songs to form a playlist for a new user (\ie \emph{cold user}).
%We find it is challenging to improve recommendations beyond simply ranking songs according to their popularity,
%which is consistent with discoveries in~\cite{mcfee2012million,bonnin2013evaluating,bonnin2015automated}.
% the reason is believed to be that a small number of popular songs appeared in a large number of playlists, 
% and the majority of songs only appear in a few playlists~\cite{bonnin2013evaluating}.
% This is consistent with discoveries in~\cite{bonnin2013evaluating,jannach2015beyond,bonnin2015automated}

We formulate playlist recommendation as a multitask learning problem.
Firstly, we study the setting of recommending %a set of songs to form %a new playlist for a user
personalised playlists for a user
%We learn the preference of the given user from her existing playlists,
by exploiting the (implicit) preference %of the given user 
from her existing playlists.
Since we do not have any contextual information about the new playlist, 
we call this setting \emph{cold playlists}.
We find that learning from a user's existing playlists %can significantly 
improves the accuracy of recommendation compared to 
suggesting popular songs from familiar artists.
%sophisticated baselines that employ song, artist or user information.
%
We further consider the setting of \emph{cold users} (\ie new users),
where we recommend playlists for new users %by learning from playlists of existing users
given playlists from existing users.
We find it challenging to improve recommendations beyond simply ranking songs according to their popularity 
if we know nothing except the identifier of the new user, %about the user,
which is consistent with previous 
discoveries~\cite{mcfee2012million,bonnin2013evaluating,bonnin2015automated}.
However, improvement can still be achieved if we know a few simple attributes (\eg age, gender, country) % etc.)
of the new users.
%
%This motivates us to exploit the (implicit) user preference from her existing playlists,
%which leads to the \emph{cold playlists} setting,
%where a set of songs are recommended to form a new playlist for an existing user. %of a music service.
%We find that learning from a user's existing playlists %can significantly 
%improves the accuracy of recommendation.
%
%In addition, 
%We further investigate the value of seed songs for improving recommendation.
%This results in the \emph{cold songs} setting,
%where we recommend a set of newly released songs to extend users' existing playlists. %from users.
%
Lastly, we investigate the setting of recommending newly released songs (\ie \emph{cold songs}) 
to extend users' existing playlists. 
%We call this setting \emph{cold songs}.
We find that the set of songs in a playlist are particularly %especially 
helpful in guiding %the process of adding new songs to the given playlist
the selection of new songs to be added to the given playlist.

% The challenge of this task, besides those brought by cold-start settings, is two-fold:
% \begin{enumerate}[(i)]
%	\item \emph{Sparsity}. While a large number of playlists are hosted by music streaming services,
% playlists for each user is still limited. This is exacerbated by the long-tailed distribution that
% most users only have a few playlists, while a small portion of users have a very large number of playlists.
% Similarly, a small number of popular songs appeared in a large number of playlists, and the majority of songs
% only appear in a few playlists~\cite{bonnin2013evaluating}.
%	\item \emph{Noise}. There could be noise in metadata~\cite{bonnin2015automated} or users might randomly choose songs 
% from their music collection when composing playlists~\cite{mcfee2012hypergraph}.
%% which makes it hard to learn the intent of a playlist.
% \end{enumerate}
%These challenges make it hard to make recommendations better than simply ranking songs according to their 
% popularity~\cite{mcfee2012million,bonnin2013evaluating,bonnin2015automated},



%%In this paper, we propose a novel approach which ranks 
% In this paper, we propose a novel multitask learning method for playlist recommendation in cold-start settings.
We propose a novel multitask learning method that %to %which can 
%deals with 
can deal with %handle
playlist recommendation in all three cold-start settings.
It optimises a bipartite ranking loss~\cite{Freund:2003,Agarwal:2005}
that encourages songs in a playlist to be ranked higher than those that are not.
This results in a convex optimisation problem with an enormous number of constraints.
Inspired by an equivalence between bipartite ranking and binary classification,
we efficiently approximate an optimal solution of the constrained objective 
by minimising an unconstrained classification loss.
%all three cold-start settings for playlist recommendation.
%It aims to 
%It optimises a bipartite ranking loss that 
%encourages songs that are likely to be in a playlist to be ranked higher than those unlikely to be chosen. 
%encourages ranking songs that are likely to be in a playlist above those unlikely to be chosen. 
%
%It learns to rank songs that are likely to be in a playlist %that could end up being in a playlist %for the given user 
%above those unlikely to be chosen by minimising a bipartite ranking loss.
%
%To learn the parameters, 
%We optimise the multitask learning objective by formulating %solving 
%which involves 
%a convex constrained optimisation problem.
%To avoid dealing with an enormous number of constraints,
%we approximate an optimal solution %of the multitask learning objective
%by minimising an unconstrained objective 
%inspired by an equivalence %relationship 
%between bipartite ranking and binary classification.
%
%we approximate an optimal solution of the multitask learning objective by minimising an unconstrained classification loss.
%we show how one can approximate the multitask objective to avoid dealing with a large number of constraints.
%using an equivalence relationship between bipartite ranking and binary classification.
%This results in an unconstrained objective that approximates the multitask objective.
%
We present experiments on two real playlist datasets, %for all three cold-start settings,
and demonstrate that our multitask learning approach improves over existing strong baselines 
for playlist recommendation in cold-start scenarios.
%
%This approach achieves good performance in cold-start playlist recommendation on two real playlist datasets.
%
%Experiments on two real playlist datasets show the proposed approach 
%has good performance for playlist recommendation in cold-start scenarios. %settings.
%and the classification approach in particular significantly improves the learning efficiency 
%while maintaining the same level of high performance as the ranking approach.
%
% Our paper is organised as follows:
% Section 2 discusses previous study that are most relevant to our work.
%%Section 2 illustrates the three cold-start settings we investigate and summarises recent related work.
%Section 3 describes the multitask objective and approaches to optimise it.
%%a bipartite ranking loss which aims to ranks songs in a playlist higher than those are not.
%%We optimises the multitask objective by solving a constrained optimisation problem.
%%We then describe a classification loss which enables us to approximately optimise the objective by
%%solving an unconstrained optimisation problem.
%%Section 4 discusses previous work that are most relevant to our work.
% Section 4 details experiments of cold-start playlist recommendation. % on two real playlist datasets.
% Lastly, we summarises the paper and describes future work in Section 5.
