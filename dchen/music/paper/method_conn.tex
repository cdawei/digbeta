\subsection{Relationship between the classification method and bipartite ranking method}

% one paragraph describing the relationship between binary classification and bipartite ranking

The two methods proposed in this section is ostensibly different, 
however, they are in fact closely related, we detail this relationship in this section.

Suppose we approximate the \emph{max} operator
\begin{equation*}
\max_i z_i \approx \frac{1}{p} \log \sum_i e^{p z_i},
\end{equation*}
and use the exponential surrogate $\ell(f, y) = e^{-fy}$, we have
\begin{equation*}
\begin{aligned}
&\ell \left( (\bv_{u(i)} + \w_i + \mubm)^\top \phibm_{u(i),m} - \max_{n: y_n^i = 0} (\bv_{u(i)} + \w_i + \mubm)^\top \phibm_{u(i),n} \right) \\
&\approx e^{-(\bv_{u(i)} + \w_i + \mubm)^\top \phibm_{u(i),m}}
   \left( \sum_{n: y_n^i = 0} e^{p (\bv_{u(i)} + \w_i + \mubm)^\top \phibm_{u(i),n}} \right)^\frac{1}{p}.
\end{aligned}
\end{equation*}

The empirical risk of the bipartite ranking formulation with the exponential surrogate 
can then be approximated as
\begin{equation*}
\RCal_\textsc{rank} 
\approx \widetilde\RCal_\textsc{rank} 
= \frac{1}{N} \sum_{i=1}^N \frac{1}{M_+^i} 
  \left( \sum_{m: y_m^i = 1} e^{-f(u(i), i, m)} \right)
  \left( \sum_{n: y_n^i = 0} e^{p f(u(i), i, n)} \right)^\frac{1}{p}.
\end{equation*}

Recall the empirical risk of the classification problem is
\begin{equation*}
\RCal_\textsc{clf} 
= \frac{1}{N} \sum_{i=1}^N \left( 
  \frac{1}{M_+^i} \sum_{m: y_m^i = 1} e^{-f(u(i), i,m)} 
  + \frac{1}{M_-^i p} \sum_{n: y_n^i = 0} e^{p f(u(i), i,n)} \right),
\end{equation*}

Let $\thetabm$ denote all parameters $\V, \W, \mubm$, we have the following theorem:
\begin{theorem}
\label{th:rank2clf}
Problem $\min_{\thetabm} \RCal_\textsc{clf}$ and $\min_{\thetabm} \widetilde\RCal_\textsc{rank}$ share the same minimisers
(assuming minimisers exist).
\end{theorem}

\begin{proof}
We start by introducing a constant bias feature $x_m^0 = 1$ for all examples $m \in \{1,\dots,M\}$.
then we show the minimisers of one problem also minimise the other, and vice versa.
\end{proof}
