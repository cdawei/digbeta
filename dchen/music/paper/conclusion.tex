\section{Conclusion and future work}

We study the problem of recommending playlists to users in three cold-start settings:
cold playlists, cold users and cold songs.
We propose a multitask learning method that learns user- and playlist-specific weights 
as well as shared weights from user-curated playlists,
which allows us to form new personalised playlists for an existing user, %in the cold playlists setting,
produce playlists for a new user, %in the cold users setting, 
and extend users' playlists with newly released songs. %in the cold songs setting.
We optimise the parameters of the multitask learning method by minimising a bipartite ranking loss
that encourages songs in a playlist to be ranked higher than those are not,
and an equivalence between bipartite ranking and binary classification further enables the efficient 
approximation of optimal parameters.
Empirical evaluations on two real playlist datasets demonstrate the effectiveness of the proposed
method in recommending playlists in cold-start settings.
%
%We investigate the problem of recommending playlists to users in cold-start settings.
%In the {\it cold songs} setting, we recommend newly released songs to extend existing playlists;
%in the {\it cold playlists} setting, we recommend a set of songs to form a new playlist for an existing user;
%and in the {\it cold users} setting, we recommend a set of songs to form a playlist for a new user.
%We deal with all three cold-start settings using a multitask learning method which encourages songs in playlist 
%to be ranked higher than those are not by minimising a bipartite ranking loss. 
%We formulate the multitask objective as a constrained convex optimisation problem, and further approximates it 
%by an unconstrained objective inspired by an equivalence relationship between bipartite ranking and binary classification. 
%Empirical results on two real music playlist datasets show the proposed approach 
%has good performance for playlist recommendation in cold-start settings.
%
%We are aware of the limitations of our proposed approach, 
% Our proposed approach has a few limitations,
% which we leave as future work.
% In particular, 
For future work, we would like to explore 
auxiliary data sources (\eg music information shared on social media) and additional features for songs and users 
(\eg lyrics, user profiles), % as well as the sequential order of songs in playlist %which could 
%provide additional information to 
which could potentially help make better recommendations.
Further, non-linear models such as deep neural networks have been shown to work extremely well in a wide range of tasks,
and the linear model with sparse parameters in this work would likely be more compact if non-linear objective were adopted.

% Finally, as a remark, we want to mention the challenge of evaluating the recommended results.
% While metrics in information retrieval are commonly employed, recommender system is more like a generative process
% than a information retrieval task. Fortunately, this challenge has been noticed and been attacked in a number of
% ways~\cite{mcfee2011natural,mcfee2012hypergraph,schedl2017}, 
% we believe that promising automatic evaluation methods that accepted by the (majority of) 
% community is one premise of significant progress in music recommendation.
