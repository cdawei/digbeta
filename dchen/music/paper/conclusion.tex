\section{Conclusion}


We propose a flexible multitask learning method to deal with all
three settings. The method learns from user-curated playlists, and
encourages songs in the playlist to be ranked higher than those
are not by minimising a bipartite ranking loss. We formulate the
objective as a constrained convex optimisation problem, and show
how this may be approximated by an unconstrained objective in-
spired by an equivalence relationship between bipartite ranking and
binary classification. Empirical results on two real music playlist
datasets show the proposed approach has good performance for
playlist recommendation in cold-start settings.


We investigate the problem of recommending playlists to users in cold-start settings.
In the cold songs setting, we recommend newly released songs to extend existing playlists;
in the cold playlists setting, we recommend a set of songs to form a new playlist for an existing user;
and in the cold users setting, we recommend a set of songs to form a new playlist for a new user.
We deal with all three settings using a multitask learning method which encourages songs in playlist 
to be ranked higher than those are not by minimising a bipartite ranking loss. 
We formulate the objective as a constrained convex optimisation problem, and further approximates it 
by an unconstrained objective inspired by an equivalence relationship between bipartite ranking and
binary classification. 
Empirical results on two real music playlist datasets show the proposed approach 
has good performance for playlist recommendation in cold-start settings.

We are aware of a few limitations of the proposed approach, which we leave as future work.
Specifically, additional data sources (\eg music information shared on social media) or song/user 
features (\eg lyrics, user profile), as well as the sequential order of songs which could provide 
additional information to help make better recommendations.
Further, non-linear models such as deep neural networks have shown strong performance in a wide arrange of tasks,
and the linear model with sparse parameters in this work could potentially be more compact if non-linear objective were employed.

Finally, as a remark, we want to mention the challenge of evaluating the recommended results.
While metrics in information retrieval are commonly used, recommender system is more like a generative process
than a information retrieval task. Fortunately, this challenge has been noticed and been attacked in many 
ways~\cite{mcfee2011natural,mcfee2012hypergraph,schedl2017}, 
we believe that promising automatic evaluation methods that accepted by the (majority of) 
community is one premise of significant progress in music recommendation.
