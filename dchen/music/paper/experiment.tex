\section{Experiment}
\label{sec:experiment}

We first evaluate the proposed method on the task of tagging text, 
using two standard multi-label classification dataset~\cite{katakis2008multilabel}.

\subsection{Multi-label classification}

Performance on multi-label dataset\footnote{Results of PRLR are taken from~\citep{lin2014multi}.}
are shown in terms of F$_1$ scores averaged over both examples and labels, 
as well as precision at $k$ which focus on the top-$k$ predictions.

The results are summarised in Table~\ref{tab:perf_mlc},
where the multi-label p-classification method output performance the binary relevance baseline by a large margin.
Further, it achieves better performance than (\citet{lin2014multi}) which encourages sparse and low-rank predictions.
Finally, it is encouraging that our method performs better than (\citet{belanger2016structured}) and (\citet{gygli2017deep}),
both work learn complex non-linear functions using deep neural networks, compared with our method which uses a linear function.

\begin{table}[!h]
\centering
\caption{Performance on multi-label dataset}
\label{tab:perf_mlc}
\resizebox{\linewidth}{!}{
\setlength{\tabcolsep}{2pt} % tweak the space between columns
%\begin{tabular}{l*{6}{c}}
\begin{tabular}{l|ccc|ccc}
\toprule
{} & \multicolumn{3}{c|}{\textbf{bibtex}} & \multicolumn{3}{c}{\textbf{bookmarks}} \\
{} &   F$_1$ Example & F$_1$ Label &    AUC &      F$_1$ Example & F$_1$ Label &    AUC \\
\midrule
Binary Relevance~\cite{}           &          $37.9$ &      $30.1$ & $85.3$ &             $29.5$ &      $21.0$ & $87.2$ \\
PRLR~\cite{lin2014multi}           &          $44.2$ &      $37.2$ &    N/A &             $34.9$ &      $23.0$ &    N/A \\
SPEN~\cite{belanger2016structured} &          $41.3$ &      $33.7$ & $92.6$ &             $35.5$ &      $24.1$ & $90.8$ \\
DVN~\cite{gygli2017deep}           &          $44.7$ &      $32.4$ & $86.7$ &             $37.2$ &      $23.7$ & $76.9$ \\
MLR (Ours)                         &          ${\bf 47.0}$ & ${\bf 38.8}$ & ${\bf 93.3}$ & ${\bf 37.7}$ & ${\bf 28.4}$ & ${\bf 91.8}$ \\
\bottomrule
\end{tabular}
}
\end{table}


\subsection{Playlist augmentation}

We further evaluate our proposed approach on a music playlist dataset, 
where we focus on the task of augmenting user created playlist.

In particular, we split the songs in AotM-2011~\cite{mcfee2012hypergraph} playlist dataset
according to the ratio 70\%:10\%:20\%, which results in the training, validation and test set, respectively.
Given a new song which does not appeared in any playlist, 
we predict, for each playlist, whether to augment it with this song.

\newpage

\subsubsection{A few conclusions on a subset}
%\begin{table}[!h]
%\centering
%\caption{Statistics of subset}
Statistics of subset \\
\resizebox{\linewidth}{!}{
\begin{tabular}{cccccc}
\toprule
\#Users & \#Songs & \#Playlists (all)  & \#Playlists (incomplete) & \#Unknown entries &\#Song Features (audio + genre) \\
\midrule
15      & 786     & 108                & 24   & 18,723  & 217 \\
\bottomrule
\end{tabular}
}
%\end{table}


\paragraph{Which type of loss is most helpful?}
%\begin{table}[!h]
%\centering
%\caption{Empirical results on subset}
Empirical results on subset \\
\resizebox{\linewidth}{!}{
\begin{tabular}{l|cccc}
\toprule
{}            & Row-wise Loss & Column-wise Loss & Row-wise + Column-wise loss & Independent Logistic Regression \\
\midrule
Run 1         & 0.5495              & 0.5373           & 0.5404     & 0.5373 \\
Run 2         & 0.5030              & 0.5040           & 0.5017     & 0.4426 \\
Run 3         & 0.4509              & 0.4531           & 0.4522     & 0.4916 \\
Run 4         & 0.5396              & 0.5485           & 0.5459     & 0.5142 \\
Run 5         & 0.5376              & 0.5670           & 0.5615     & 0.5042 \\
Mean          & 0.5161              & 0.5220           & 0.5203     & 0.0440 \\
Std           & 0.0405              & 0.0448 
\bottomrule
\end{tabular}
}
%\end{table}

\paragraph{Experimental design:}
C: 1, 1, 1, p: 1, no multi-task regularisation


\paragraph{Is multi-task regularisation helpful?}
%\begin{table}[!h]
%\centering
%\caption{Empirical results on subset}
Empirical results on subset \\
\resizebox{\linewidth}{!}{
\begin{tabular}{l|ccc}
\toprule
{}            & Multi-task Reg. + Row-wise Loss & Multi-task reg. + Column-wise Loss & Multi-task reg. + Row-wise + Column-wise Loss  \\
\midrule
Run 1         & 0.5783              & 0.5941     & 0.5886  \\
Run 2         & 0.5272              & 0.5390     & 0.5328  \\
Run 3         & 0.6452              & 0.6224     & 0.6393  \\
Run 4         & 0.6163              & 0.6229     & 0.6232  \\
Run 5         & 0.6561              & 0.6278     & 0.6483  \\
Mean          & 0.6046              & 0.6012     & 0.6064  \\ 
Std           & 0.0527              & 0.0372     & 0.0470  \\
\bottomrule
\end{tabular}
}
%\end{table}

\paragraph{Experimental design:}
C: 1, 1, 1, p: 1, with multi-task regularisation, no user specific regularisation parameter.


\paragraph{Is user-specific multi-task regularisation parameter (based on \#playlists) helpful?}
%\begin{table}[!h]
%\centering
%\caption{Empirical results on subset: user-specific regularisation parameter = 1/\#playlists}
Empirical results on subset: user-specific regularisation parameter = 1/\#playlists \\
\resizebox{\linewidth}{!}{
\begin{tabular}{l|ccc}
\toprule
{}            & Multi-task Reg. + Row-wise Loss + user-spec reg. & Multi-task reg. + Column-wise Loss + user-spec reg. & Multi-task reg. + Row-wise + Column-wise Loss + user-spec reg. \\
\midrule
Run 1         & 0.5652       & 0.5340           & 0.5470 \\
Run 2         & 0.5940       & 0.5676           & 0.5772 \\
Run 3         & 0.5661       & 0.5181           & 0.5315 \\
Run 4         & 0.6159       & 0.5566           & 0.5790 \\
Run 5         & 0.5449       & 0.5458           & 0.5460 \\
Mean          & 0.5772       & 0.5444           & 0.5561 \\
Std           & 0.0278       & 0.0193           & 0.0210 \\
\bottomrule
\end{tabular}
}
%\end{table}


\newpage


\subsection{New song recommendation}

\subsubsection{A few conclusions on a subset}
%\begin{table}[!h]
%\centering
%\caption{Statistics of subset}
Statistics of subset \\
\resizebox{\linewidth}{!}{
\begin{tabular}{cccccc}
\toprule
\#Users & \#Songs (train/test) & \#Playlists & \#Unknown entries &\#Song Features (audio + genre) \\
\midrule
15      & 629 / 157  & 108 & 16,956  & 217 \\
\bottomrule
\end{tabular}
}
%\end{table}

\paragraph{Which type of loss is most helpful?}
%\begin{table}[!h]
%\centering
%\caption{Empirical results on subset}
Empirical results on subset \\
\resizebox{\linewidth}{!}{
\begin{tabular}{l|cccc}
\toprule
{}            & Row-wise Loss & Column-wise Loss & Row-wise + Column-wise loss & Independent Logistic Regression \\
\midrule
Run 1         & 0.6395              & 0.6448           & 0.6431     & 0.6022 \\
Run 2         & 0.6474              & 0.6565           & 0.6567     & 0.6197 \\
Run 3         & 0.6172              & 0.6144           & 0.6168     & 0.5366 \\
Run 4         & 0.6741              & 0.6840           & 0.6818     & 0.6067 \\
Run 5         & 0.6506              & 0.6449           & 0.6498     & 0.5716 \\
Mean          & 0.6458              & 0.6489           & 0.6496     & 0.5874 \\
Std           & 0.0205              & 0.0251           & 0.0235     & 0.0334 \\
\bottomrule
\end{tabular}
}
%\end{table}

\paragraph{Experimental design:}
C: 1, 1, 1, p: 1, no multi-task regularisation.



\paragraph{Is multi-task regularisation helpful?}

%\begin{table}[!h]
%\centering
%\caption{Empirical results on subset}
Empirical results on subset \\
\resizebox{\linewidth}{!}{
\begin{tabular}{l|ccc}
\toprule
{}            & Multi-task Reg. + Row-wise Loss & Multi-task reg. + Column-wise Loss & Multi-task reg. + Row-wise + Column-wise Loss  \\
\midrule
Run 1         & 0.6543        & 0.6358           & 0.6472 \\
Run 2         & 0.6327        & 0.6572           & 0.6482 \\
Run 3         & 0.6633        & 0.6722           & 0.6709 \\
Run 4         & 0.6267        & 0.6765           & 0.6615 \\
Run 5         & 0.5967        & 0.6104           & 0.6062 \\
Mean          & 0.6347        & 0.6504           & 0.6468 \\
Std           & 0.0260        & 0.0274           & 0.0247 \\
\bottomrule
\end{tabular}
}
%\end{table}

\paragraph{Experimental design:}
C: 1, 1, 1, p: 1, with multi-task regularisation, no user specific regularisation parameter.



\paragraph{Is user-specific multi-task regularisation parameter (based on \#playlists) helpful?}
%\begin{table}[!h]
%\centering
%\caption{Empirical results on subset: user-specific regularisation parameter = 1/\#playlists}
Empirical results on subset: user-specific regularisation parameter = 1/\#playlists \\
\resizebox{\linewidth}{!}{
\begin{tabular}{l|ccc}
\toprule
{}            & Multi-task Reg. + Row-wise Loss + user-spec reg. & Multi-task reg. + Column-wise Loss + user-spec reg. & Multi-task reg. + Row-wise + Column-wise Loss + user-spec reg. \\
\midrule
Run 1         & 0.6714             & 0.6744     & 0.6777 \\
Run 2         & 0.6560             & 0.6706     & 0.6646 \\
Run 3         & 0.6592             & 0.6641     & 0.6611 \\
Run 4         & 0.6444             & 0.6528     & 0.6493 \\
Run 5         & 0.6558             & 0.6648     & 0.6627 \\
Mean          & 0.6573             & 0.6653     & 0.6631 \\
Std           & 0.0097             & 0.0082     & 0.0101 \\
\bottomrule
\end{tabular}
}
%\end{table}
