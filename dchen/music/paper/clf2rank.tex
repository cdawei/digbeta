\section{Proof of classification objective minimises ranking objective}

The empirical risk $\RCal_\textsc{rank}$ can be approximated (with the exponential surrogate) as follows:
\begin{equation*}
\begin{aligned}
\RCal_\textsc{rank} 
&= \frac{1}{N} \sum_{i=1}^N \frac{1}{M_-^i} \exp \left( -\min_{m: y_m^i = 1} f(i, m) \right) \sum_{n: y_n^i = 0} \exp(f(i, n)) \\
&\approx \frac{1}{N} \sum_{i=1}^N \frac{1}{M_-^i} \exp \left( \frac{1}{p} \log\sum_{m: y_m^i = 1} e^{-p f(i, m)} \right)
         \sum_{n: y_n^i = 0} e^{f(i, n)} \\
&= \frac{1}{N} \sum_{i=1}^N \frac{1}{M_-^i} \exp \left( \log \left( \sum_{m: y_m^i = 1} e^{-p f(i, m)} \right)^\frac{1}{p} \right) 
   \sum_{n: y_n^i = 0} e^{f(i, n)} \\
&= \frac{1}{N} \sum_{i=1}^N \frac{1}{M_-^i} \left( \sum_{m: y_m^i = 1} e^{-p f(i, m)} \right)^\frac{1}{p} \sum_{n: y_n^i = 0} e^{f(i, n)} \\
&= \widetilde\RCal_\textsc{rank}.
\end{aligned}
\end{equation*}

Let $\RCal_\textsc{clf}$ be the following classification risk:
\begin{equation*}
\RCal_\textsc{clf} 
= \frac{1}{N} \sum_{i=1}^N \left( 
  \frac{1}{p M_+^i} \sum_{m: y_m^i = 1} e^{-p f(i, m)} 
  + \frac{1}{M_-^i} \sum_{n: y_n^i = 0} e^{f(i, n)} \right),
\end{equation*}
we have a theorem
\begin{theorem}
\label{th:rank2clf}
Let $\uptheta^* \in \argmin_{\uptheta} \RCal_\textsc{clf}$ (assuming minimisers exist), 
then $\uptheta^* \in \argmin_{\uptheta} \widetilde\RCal_\textsc{rank}$.
\end{theorem}

\begin{proof}
We following the proof technique in~\cite{}.
First, we introduce a constant feature $1$ for $x_m, \, m \in \{1,\dots,M\}$ to serve as a bias, 
and without loss of generality, let $x_m^0$ be the constant feature, \ie $x_m^0 = 1$, 
$\forall i \in \{1,\dots,N\}$, let
\begin{equation*}
\begin{aligned}
0 
&= \frac{\partial \RCal_\textsc{clf}} {\partial v_i^0}
&= \frac{1}{p M_+^i} \sum_{m: y_m^i = 1} e^{-p f(i, m)} (-p) + \frac{1}{M_-^i} \sum_{n: y_n^i = 0} e^{f(i, n)},
\end{aligned}
\end{equation*}
we have
\begin{equation}
\label{eq:eq1}
\frac{1}{M_+^i} \sum_{m: y_m^i = 1} e^{-p f(i, m)} = \frac{1}{M_-^i} \sum_{n: y_n^i = 0} e^{f(i, n)}.
\end{equation}

Further, let
\begin{equation*}
0 
= \frac{\partial \RCal_\textsc{clf}} {\partial \bv_i} 
= \frac{1}{p M_+^i} \sum_{m: y_m^i = 1} e^{-p f(i, m)} (-p \x_m) + \frac{1}{M_-^i} \sum_{n: y_n^i = 0} e^{f(i, n)} \x_n,
\end{equation*}
we have
\begin{equation}
\label{eq:eq2}
\frac{1}{M_+^i} \sum_{m: y_m^i = 1} e^{-p f(i, m)} \x_m = \frac{1}{M_-^i} \sum_{n: y_n^i = 0} e^{f(i, n)} \x_n.
\end{equation}

We want to prove that
$\frac{\partial \widetilde\RCal_\textsc{rank}} {\partial \bv_i}$ equals to $0$, note that
\begin{equation*}
\begin{aligned}
&\frac{\partial \widetilde\RCal_\textsc{rank}} {\partial \bv_i} \\
&= \frac{1}{N M_-^i} \left[ \frac{1}{p} 
   \left( \sum_{m: y_m^i=1} e^{-p f(i, m)} \right)^{\frac{1}{p} - 1} \sum_{m: y_m^i=1} e^{-p f(i, m)} (-p) \x_m \sum_{n: y_n^i=0} e^{f(i, n)} \right. \\
& \hspace{4em} \left.
   + \left( \sum_{m: y_m^i = 1} e^{-p f(i, m)} \right)^\frac{1}{p} \sum_{n: y_n^i = 0} e^{f(i, n)} \x_n \right] \\
&= \frac{-1}{N M_-^i} \left( \sum_{m: y_m^i=1} e^{-p f(i, m)} \right)^{\frac{1}{p} - 1} 
   \left[ \sum_{m: y_m^i=1} e^{-p f(i, m)} \x_m \sum_{n: y_n^i=0} e^{f(i, n)} \right. \\
& \hspace{4em} \left.
   - \sum_{m: y_m^i = 1} e^{-p f(i, m)} \sum_{n: y_n^i = 0} e^{f(i, n)} \x_n \right] \\
&= \frac{-1}{N M_-^i} \left( \sum_{m: y_m^i=1} e^{-p f(i, m)} \right)^{\frac{1}{p} - 1} 
   \left[ \sum_{m: y_m^i=1} e^{-p f(i, m)} \x_m \frac{M_-^i}{M_+^i} \sum_{m: y_m^i=1} e^{-p f(i, m)} \right. \\
& \hspace{4em} \left.
   - \sum_{m: y_m^i = 1} e^{-p f(i, m)} \sum_{n: y_n^i = 0} e^{f(i, n)} \x_n \right] \ \text{(by Eq.~\ref{eq:eq1})} \\
&= \frac{-1}{N M_-^i} \left( \sum_{m: y_m^i=1} e^{-p f(i, m)} \right)^\frac{1}{p} 
   \left[ \frac{M_-^i}{M_+^i} \sum_{m: y_m^i=1} e^{-p f(i, m)} \x_m - \sum_{n: y_n^i = 0} e^{f(i, n)} \x_n \right] \\
&= 0. \ \text{(by Eq.~\ref{eq:eq2})}
\end{aligned}
\end{equation*}

We can similarly prove that 
\begin{equation*}
\begin{aligned}
\frac{\partial \widetilde\RCal_\textsc{rank}}{\partial \bu_j} &= 0, \ \forall j \in \{1,\dots,U\} \\
\frac{\partial \widetilde\RCal_\textsc{rank}}{\partial \widebar{\bu}} &= 0.
\end{aligned}
\end{equation*}

\end{proof}


