% !TEX root=main.tex

We formalised the problem of eliminating loops when recommending trajectories to visitors in a city,
and surveyed three distinct approaches to the problem --
%a heuristic inspired by Christofides' algorithm, list extensions of the Viterbi algorithm, and integer linear programming.
graph-based heuristics,
list extensions of the Viterbi algorithm,
and integer linear programming.
We explicated how two ostensibly different approaches to the list Viterbi algorithm \citep{seshadri1994list,nilsson2001sequentially} are in fact fundamentally identical.

In experiments on real-world datasets,
a greedy graph-based heuristic offered excellent performance and runtime.
We thus recommend its use
for removing loops at prediction time
over the more involved integer programming and list Viterbi algorithms.

As a caveat on the applicability of the greedy algorithm,
we note that the problem of removing loops also arises during training SSVMs in the loss-augmented inference step \citep{Chen:2017}.
The list Viterbi algorithm has been demonstrated useful in this context;
it is unclear whether the same will be true of the approximate greedy algorithm, as it will necessarily lead to sub-optimal solutions.

As future work, it is of interest to extend the greedy algorithm to the top-$K$ evaluation setting of \citet{Chen:2017}, wherein the recommender produces a \emph{list} of paths to be considered.
A natural strategy would be to augment the algorithm with a beam search.

Further, the idea of modifying the standard Viterbi inference problem (Equation \ref{eqn:argmax}) has other applications, such
as ensuring diversity in the predicted ranking.
Such problems have been studied in contexts such as information retrieval \citep{Carbonell:1998} and computer vision \citep{Park:2011},
and their study would be interesting in trajectory recommendation.
More broadly, investigation of efficient means of ensuring global cohesion -- \eg preventing homogeneous results -- 
is an important direction for the advancement of citizen-centric recommendation.

% {\color{red!75}
% \begin{itemize}
% 	\item Learning
% 	\item diversity, MMR
% 	\item See whether any of the workshop topics might have problems that this paper applies.
% \end{itemize}
% }
