% !TEX root=./main.tex

\secmoveup
\section{Related work}
\label{sec:related}
\textmoveup

We contrast the problem setting and approach of this paper to a number of sub-fields in the recommender systems and machine learning literature.

% \secmoveup
% \subsection{Recommender systems}
% \textmoveup

\textbf{Recommender systems}.
There is a rich body of work on recommender systems,
with the
Netflix prize~\citep{Netflix} encapsulating the canonical problem that concerns most research in this field:
the recommendation of \emph{static} content such as books or movies from some large database~\citep{Goldberg:1992,Sarwar:2001}.
Matrix factorisation methods have proven particularly effective for such problems~\citep{Koren:2009}.
%The latter are a prototypical example of a \emph{collaborative filtering} approach to recommendation,
%wherein one exploits the ``wisdom of the crowd'' implicit in the preferences of several users.

Standard matrix factorisation methods
have been extended to tackle
diverse problem settings such as
time-varying preferences~\citep{Koren:2009b} and implicitly provided feedback~\citep{Rendle:2009}.
However,
the setting of recommending \emph{structured content},
such as trajectories as considered in this paper,
has received less attention but for a few exceptions (see below).
There are clear challenges with applying matrix factorisation techniques in our problem.
First, most users are only associated with a single trajectory, which defeats any hope of inferring a complex preference embedding for them.
Second, even if one had access to multiple trajectories per user, it is unclear how to find a latent embedding for entire trajectories,
as there are exponentially many possible values for the latter.


% \secmoveup
% \subsection{Structured content recommendation}
% \textmoveup

\textbf{Structured content recommendation}.
The recommendation of structured content has
been studied in three distinct subfields.
The first is work on recommending the next item a user might like to purchase, given the sequence of their shopping basket purchases~\citep{Rendle:2010,Wang:2015}.
The canonical approach here is to apply the matrix factorisation idea to the Markov chain of transitions between items.
The success of this method relies on the fact that one is only interested in predicting sequences one element at a time, so that latent embeddings for each item may be found.

The second is work on recommending song playlists to users, given a query song~\citep{McFee:2011,chen2012playlist}.
The canonical approach here is to
learn a latent representation of songs from historical playlist data,
and exploit a Markovian assumption on the song transitions.
While a reasonable first order approximation, this assumption limits the modelling power of such approaches.

The third is prior work on travel route recommendation, which we survey in detail.


% \secmoveup
% \subsection{Travel route recommendation}
% \textmoveup

\textbf{Travel route recommendation}.
Recommendation problems involving travel routes have received considerable interest of late~\cite{bao2015recommendations,zheng2015trajectory,zheng2014urban}.
There are, roughly, three problem settings that have been studied.
The first setting is \emph{point of interest} (\emph{POI}) \emph{recommendation}.
Here, one wishes to rank various POIs in a city in order of how ``interesting'' they are to a given visitor,
exploiting
available metadata for each POI.
Typically, this problem is tackled via
collaborative filtering on user-location affinity~\cite{shi2011personalized,lian2014geomf,hsieh2014mining,yuan2014graph}.

The second setting is \emph{next location recommendation}.
Here, given the sequence of a traveller's partial tour through a city,
the goal is to recommend which POI the traveller should visit next.
This can be understood as a variant of POI recommendation with strong contextual information provided.
Typically, this problem is tackled via
incorporating Markov chains into collaborative filtering~\cite{fpmc10,ijcai13,zhang2015location}.
%quantifying tourist traffic flow between points-of-interest~\cite{zheng2012patterns},
%formulating a binary decision or ranking problem~\cite{baraglia2013learnext}, and predicting the next location with
%or sequence models such as recurrent neural networks~\cite{aaai16}.
The third setting is trajectory recommendation,
which is our focus in this paper.
Typically, this problem is tackled via
a heuristic combination of locations and routes~\cite{lu2010photo2trip,ijcai15,lu2012personalized}, or
by solving an optimisation problem that does not exploit historical data~\cite{gioniswsdm14,chen2015tripplanner}.


% \secmoveup
% \subsection{Learning to rank}
% \textmoveup
% AKM: probably can omit this stuff as I dunno how to frame it properly

% \textbf{Learning to rank}.
% The sequence recommendation problem can be related at an abstract level to the label ranking problem~\citep{Dekel:2003},
% where the input comprises a query and a corresponding graph as a label.
% The goal in such problems is to similarly learn a ranking over the nodes of the graph;
% however, these are typically not learned using structured prediction models,
% but rather by computing a loss function over pairs.

% Our problem can also be related to the listwise ranking approach in information retrieval~\citep{Cao:2007},
% which attempts to learn a good set of results for a query by exploiting structure embedded in the entire set of results.
% In this light, previous methods for sequence recommendation are akin to pointwise and pairwise approaches for ranking~\citep{Liu:2009}.
% One contribution of this work is in casting sequence recommendation as such a structured prediction problem, where one considers the information embedded in the entire sequence associated with an input.
