% !TEX root = ./main.tex

%\secmoveup
\section{Conclusion and future work}
%\textmoveup

We cast the problem of recommending trajectories as a special case of sequence recommendation,
which we solve by augmenting the structured SVMs.
This approach not only ensures global cohesion but also accounts for the existence of multiple ground truths for a given input.
We also propose a novel application of the list Viterbi algorithm to avoid predicting loops in the sequence.
%
%
%as a structured prediction problem.
%Our solution extends structured SVMs to account for two challenges:
%(1) the existence of multiple ground truths;
%(2) avoiding predicting loops in the sequence.
%The inference procedures are novel applications of the list Viterbi algorithm.
Experiments on real-world trajectory datasets show that
sequence recommendation approaches outperform existing, non-structured approaches.

Our viewpoint enables researchers to bring recent advances in structured prediction
to bear on {\trajrec} problems,
including further improving the efficiency of inference and learning.
In the other direction, techniques from recommender systems to capture latent
user- and POI-representations, in sufficiently rich domains, may
improve the predictive power of structured prediction models.

%As a final remark, we note the cast of trajectory recommendation as a sequence recommendation problem
%is equally applicable to other applications.
%Further, the assumption that the output $\y$ is a sequence does not limit the generality of our approach,
%as inferring $\y$ of other structures can be achieved using corresponding inference and loss-augmented inference algorithms~\cite{joachims2009predicting},
%we leave this exploration as future work.
