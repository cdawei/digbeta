% !TEX root = ./main.tex

% \secmoveup
% \section{Trajectory Recommendation as a Structured Prediction Problem}
% \label{sec:trajrec}
% \textmoveup

% We present the trajectory recommendation problem,
% which shall serve as our canonical application of sequential recommendation.
% We first review the problem as it is typically treated in the literature,
% and then show how it may be viewed as a structured prediction problem.

% \secmoveup
% \subsection{Trajectory recommendation: standard view}
% \textbf{Trajectory recommendation: standard view}.
% \subsection{The trajectory recommendation problem}
\textbf{Trajectory recommendation}.
Travel recommendation problems involve a set of points-of-interest (POIs) $\mathcal{P}$ in a city~\cite{bao2015recommendations,zheng2015trajectory,zheng2014urban}.
% There are, roughly, three problem settings that have been studied.
% The first setting is \emph{point of interest} (\emph{POI}) \emph{recommendation};
% Here, one wishes to rank various POIs in a city in order of how ``interesting'' they are to a given visitor,
% exploiting
% available metadata for each POI~\cite{shi2011personalized,lian2014geomf,hsieh2014mining,yuan2014graph}.
% The second setting is \emph{next location recommendation};
% here, given the sequence of a traveller's partial tour through a city,
% the goal is to recommend which POI the traveller should visit next~\cite{fpmc10,ijcai13,zhang2015location}.
%quantifying tourist traffic flow between points-of-interest~\cite{zheng2012patterns},
%formulating a binary decision or ranking problem~\cite{baraglia2013learnext}, and predicting the next location with
%or sequence models such as recurrent neural networks~\cite{aaai16}.
%The third setting is \emph{trajectory recommendation},
%which is our interest in this paper.
%The \emph{trajectory recommendation} problem is:
Given a \emph{trajectory query} $\mathbf{x} = (s, l)$,
comprising a start POI $s \in \mathcal{P}$ and trip length
%\footnote{Instead of specifying the number of desired POIs,
%we can constrain the trajectory with a total time budget $T$.
%In this case, the number of POIs $l$ can be treated as a \emph{hidden} variable,
%with additional constraint $\sum_{k=1}^l t_k \le T$ where $t_k$ is the time spent at POI $y_k$.}
$l \!>\! 1$ (\ie the desired number of POIs, including $s$),
the \emph{trajectory recommendation} problem is to
recommend one or more sequences of POIs %%(or \emph{trajectories}) %$\mathbf{y}^*$
that maximise some notion of utility,
%To learn a suitable $f$,
%we are
%provided as input
learned from a training set
%$(\x\pb{i}, \{ \y\pb{ij} \}_{j=1:n^i})$, $i=1:n$,
%comprising pairs of queries and corresponding
of trajectories visited by travellers in the city.

% AKM: not sure these are suitable here
%
% This problem is related to automatic playlist generation,
% where we recommend a sequence of songs given a specified song (a.k.a. the seed) and the number of new songs.
% Formally, given a library of songs and a query $\mathbf{x} = (s, K)$, where $s$ is the seed and $K$ is the number of songs in playlist,
% we produce a list with $K$ songs (without duplication) by maximising the likelihood~\cite{chen2012playlist},
% \begin{equation*}
% %\max_{(y_1,\dots,y_K)} \prod_{k=2}^K \mathbb{P}(y_{k-1} \given y_k),~ y_1 = s ~\text{and}~ y_j \ne y_k,~ j \ne k.
% \mathbf{y}^* = \argmax_{\mathbf{y} \in \mathcal{P}_\mathbf{x}}~ \mathbb{P}(\mathbf{y} \given \mathbf{x}),~ \mathbf{y} = (y_1=s,\dots,y_K)
% ~\text{and}~ y_j \ne y_k ~\text{if}~ j \ne k.
% \end{equation*}

% Another similar problem is choosing a small set of photos from a large photo library and compiling them into a slideshow or movie.

%
%\subsection{Trajectory recommendation: structured view}
%\textbf{Trajectory recommendation: structured view}.
%\subsection{Trajectory recommendation as a structured prediction problem}

Existing approaches treat the problem as one of determining a score for the intrinsic appeal of each POI.
% based on implicit preference data in the training set.
% Perhaps the simplest such approach is based on learning a model to predict whether a particular POI occurs, or not, in the trajectory corresponding to a query.
% Formally, from the given set of trajectories
% we derive a training set $\{ ( \x^{(i)}, p, y^{(i)} ) \}_{i = 1}^n$,
% where each POI $p$ is augmented with a label $y^{(i)} \in \{ \pm 1 \}$ denoting whether or not it appeared in a trajectory corresponding to the query $\x^{(i)}$.
% Formally, the objective used for training is
% $$ \min_{\w} \frac{1}{2} \w^\top \w + C \cdot \sum_{i = 1}^n \sum_{p \in \mathcal{P}} \ell\left( y^{(i)} \cdot ( \w^\top \Phi( \x^{(i)}, p ) ) \right) $$
% where
% $\Phi( \cdot, \cdot )$ denotes a query-POI feature mapping,
% $C > 0$ is some regularisation strength,
% and $\ell( v )$ is any suitable margin loss, such as the logistic loss $\ell( v ) = \log( 1 + e^{-v} )$.
% One can use the resulting model to produce a score for any POI $p$ given a new query $\x$,
% and then pick the top-$l$ scoring POIs to produce a trajectory.
%%For example, \citep{cikm16paper} proposed a RankSVM model,
For example, a RankSVM model
which %is fed as input %pairs $(p_i, p_j)$ of POIs such that $p_i$ appears ahead of $p_j$ in some trajectory.
learns to predict whether a POI is likely to appear ahead of another POI in a trajectory corresponding to some query~\cite{cikm16paper}.
Formally,
from the given set of trajectories
we derive a training set
% $\{ ( \x^{(i)}, p, y^{(i)} ) \}_{i = 1}^n$,
% where each POI $p$ is augmented with a label
% %$y^{(i)} \in \{ \pm 1 \}$ denoting whether or not
% $y^{(i)}$ denoting how many times
% it appeared in a trajectory corresponding to the query $\x^{(i)}$.
$\{ ( \x^{(i)}, \mathbf{r}^{(i)} ) \}_{i = 1}^n$,
where for each trajectory query $\x^{(i)}$ there is a list of ranked POI pairs
$\mathbf{r}^{(i)} \subseteq \mathcal{P} \times \mathcal{P}$
such that
$(p, p') \in \mathbf{r}^{(i)}$
iff
%POI $p$ is ranked above the POI $p'$ in the trajectories associated with $\x^{(i)}$, %according to some notion.
%\eg based on the counts of the number of times a POI appears in the trajectories associated with a query.
POI $p$ appears more times than POI $p'$ in all trajectories associated with $\x^{(i)}$. %according to some notion.
The training objective is then
\begin{align}
\resizebox{0.7\linewidth}{!}{$\displaystyle
\displaystyle \min_{\w} ~\frac{1}{2} \w^\top \w + C \cdot \sum_{i = 1}^n \sum_{(p, p') \in \mathbf{r}^{(i)}}
\ell\left( \w^\top ( \Psi( \x^{(i)}, p ) - \Psi( \x^{(i)}, p' ) ) \right),
$} \label{eq:ranksvm}
\end{align}
for
feature mapping $\Psi$,
regularisation strength $C$ % > 0$,
and squared hinge loss $\ell( v ) = \max( 0, 1 - v )^2$.
%where $\Phi, C$ are as before,
% where
% $\Phi( \cdot, \cdot )$ denotes a query-POI feature mapping,
% $C > 0$ is some regularisation strength,
% and $\ell( v ) = \max( 0, 1 - v )^2$ is the squared hinge loss,
% and
% %$p \succeq_{\mathbf{x}} p'$ is denoted to mean that the
% $\mathcal{R}( \x )$
% denotes all pairs of POIs $(p, p')$ such that
% POI $p$ is ranked above the POI $p'$ in the trajectories associated with $\x$, %according to some notion.
% \eg based on the counts of the number of times a POI appears in the trajectories associated with a query.
%For example, this may be computed by counting the number of times a POI appears in the trajectories associated with a query,
%and using this to determine which of two POIs is more suitable for a query.

%%More abstractly,
We can view trajectory recommendation as sequence recommendation in the following way:
given trajectory query $\x$, and a suitable scoring function $f$, we wish to find
$\mathbf{y}^* = \argmax_{\mathbf{y} \in \mathcal{Y}}~f(\mathbf{x}, \mathbf{y}),$
%%DW: use top-k prediction formulation or not?
where $\mathcal{Y}$ is the set of all possible trajectories with POIs in $\mathcal{P}$ that conform to the constraints imposed by the query $\mathbf{x}$.
In particular,
$\mathbf{y} = (s,~ y_2, \dots, y_l)$ is a trajectory with $l$ POIs. %, which has no sub-tours. %i.e. $y_j \ne y_k$ if $j \ne k$.
This was the view proposed in~\cite{cikm16paper} where they authors considered an
objective function that added two components together: a POI score and a transition score.

Now, our training set of historical trajectories may be written as
$\{ ( \x\pb{i}, \{ \y\pb{ij} \}_{j=1}^{n_i} ) \}_{i=1}^{n}$,
where each $\x\pb{i}$ is a distinct query
with $\{ \y\pb{ij} \}_{j=1}^{n_i}$ the corresponding \emph{set} of observed trajectories.
Note that we expect most queries to have several distinct trajectories;
minimally,
for example,
there may be two nearby POIs that are visited in interchangeable order by different travellers.
We are also interested in predicting paths $\y$, since it is unlikely a user will want to visit the same location twice.


%
\textbf{Playlist generation}.
As another example, consider recommending song playlists (\ie sequences) to users, given a query song~\citep{McFee:2011,chen2012playlist,hidasi2015session,choi2016towards}.
A canonical approach is to
learn a latent representation of songs from historical playlists,
and exploit a Markovian assumption on the song transitions.
%While a reasonable first order approximation, this assumption limits the modelling power of such approaches.


%
\textbf{Next basket and next song prediction}.
One more example of sequence recommendation is the problem of recommending the next items a user might like to purchase, given the sequence of their shopping basket purchases~\citep{Rendle:2010,Wang:2015}.
The canonical approach here is to apply matrix factorisation to the Markov chain of transitions between items.
This method is feasible because one is only interested in predicting sequences one element at a time, instead of predicting the entire sequence given the initial item.
Recent approaches using recurrent neural networks for
next basket prediction~\cite{yu2016dynamic} and playlist
generation~\cite{choi2016towards} also focus on modelling high quality transitions only.


%
\subsection{The case for exploiting structure}

Each of the sequence recommendation problems above can be plausibly solved with approaches that do not exploit the structure inherent in the outputs $\y$. %% or do so in a simple way.
While such approaches can certainly be useful,
their modelling power is inherently limited,
as
they cannot ensure the \emph{global} cohesion of the corresponding recommendations $\y$.
%as they inherently rely on either pointwise or pairwise preferences for POIs.
For example, in the trajectory recommendation problem, the RankSVM model %as argued in Section~\ref{sec:intro},
might find three restaurants to be the highest scoring POIs;
however, it is unlikely that these form a trajectory that most travellers will enjoy.

This motivates an approach to sequence recommendation that directly ensures such global cohesion.
We now see how to do so with novel extensions to structured prediction approaches. %by attacking the problem using tools from structured prediction.

% To do this, we show how the problem can be cast as one of sequential recommendation,
% thus allowing us to leverage the tools developed in the previous section.

% Comparing the above to the discussion in the previous section, it is clear that
% we can cast trajectory recommendation as a special case of the structured recommendation problem.
% Consequently, we may approach it using structured prediction methods such as the SSVM,
% as well as the extensions proposed to account for multiple ground truths and eliminate loops during training.

% AKM: repetitive
%
% standard SSVM
% We can learn a recommender by training a SSVM on the set of observed trajectories $\{\mathbf{x}^{(i')}, \mathbf{y}^{(i')}\}_{i'=1}^{N'}$,
% However, we ignore the fact that for the same query, we normally observed more than one trajectory,
% we would like to exploit this fact to better modelling the observed trajectories.

% AKM: repetitive
%
% \subsection{Query Aggregation}
% \label{sec:query}

% To modelling the fact that a given query has multiple observed trajectories,
% we firstly group trajectories according to queries, in other words,
% we now have a dataset $\{\mathbf{x}^{(i)}, \{\mathbf{y}^{(ij)}\}_{j=1}^{N_i}\}_{i=1}^N$
% with $N$ queries and queries $\mathbf{x}^{(i)}$ has $N_i$ trajectories observed.


% \subsection{Recommendation with Multiset SSVM}
% \label{sec:trajrec-ssvm}

% We can learn to recommend trajectories by training a multiset SSVM described in Section~\ref{sec:ssvm-ms}
