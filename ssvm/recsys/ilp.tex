% !TEX root=main.tex

% \blue{
% We should probably ask Phil Kilby or Hassan about the two or three most popular subtour eliminations

% \begin{itemize}
% 	\item Elementary shortest path problems
%   %http://www.optimization-online.org/DB_FILE/2014/09/4560.pdf
% 	\item Different subtour eliminations
%   	\item (Dantzig, Fulkerson, Johnson, 1954) has 3 different versions
%   	\item (Miller, Tucker, Zemlin, 1960) has another one.
%   	\item Seems to be one more, not sure by whom (maybe Belmore, Malone, 1971)
%   %http://www.or.unc.edu/~pataki/papers/teachtsp.pdf

% \end{itemize}
% }

One way to exactly solve Equation \ref{eqn:argmax-path}
is to observe its similarity to the classical travelling salesman problem (TSP).
Indeed, if the length $l = |\PCal|$, so that every POI is restricted to be visit once,
then we exactly get the TSP problem;
however, for smaller $l$, we require visiting only a \emph{subset} of POIs,
which is not a vanilla instance of TSP.

Nonetheless, we may take inspiration from methods to solve the TSP to attack our problem.
In particular, 
we can formulate the trip recommendation problem as an \emph{integer linear program} (\emph{ILP}),
as often done for the TSP \citep{opt98}.
An off-the-shelf solver (\eg {\tt Gurobi}) may be used on the ILP.
% Instead of simply bypassing repeated revisits, 
% we can make use of subtour elimination techniques developped by the TSP research community,
% by formulating the trip recommendation problem with integer linear programming (ILP).
%The additional benefits of this formulation is that fixing the subset of POIs is not required any more,
%we can optimise over all possible subset of POIs using the off-the-shelf ILP solvers.
Formally, given a starting location $s$ and the required trip length $l$,
%we maximise the desired trip score over all possible subsets with $l$ POIs from the whole set of POIs $\{p_j\}_{j=1}^m$
we find the best possible path
via \citep{Chen:2017}

\resizebox{0.9625\linewidth}{!}{
\begin{minipage}{0.9625\linewidth}
	\begin{align*}
	\max_{\bu, \mathbf{v}} & \sum_{k=1}^m \alpha( p_k ) \cdot \sum_{j=1}^m u_{jk} +
	            \sum_{j,k=1}^m u_{jk} \cdot \beta( p_j, p_k ) \\
	s.t. 
	& \sum_{k=2}^m u_{1k} = 1, \, z(\bu)_1 = 0, z(\bu)_i \in \{ 0, 1 \} \, &&(\forall i \in \{2,\cdots,m\}) \\
	& \sum_{j,k=1}^m u_{jk} = l-1, \, \sum_{j=1}^m u_{jj} = 0, \, \sum_{j=1}^m u_{ji} \leq 1 \, &&(\forall i \in \{2,\cdots,m\}) \\
	& v_j - v_k + 1 \le (m-1) (1-u_{jk}) \, &&(\forall j,k \in \{2,\cdots,m\}).
	\end{align*}
\end{minipage}
}

\vspace{0.5\baselineskip}

Here, we index POIs such that $s = p_1$ for brevity,
and we strictly ask for a path that starts from $s$ and travels $l$ distinct POIs.
The binary $u_{jk}$ are true iff
we visit $p_k$ immediately after visiting $p_j$;
the integer $v_j$ track the rank of $p_j$;
%The binary variables $z_j$ indicate whether we end up at $p_j$.
and $z(\bu)_i \defEq \sum_{j=1}^m u_{ji} - \sum_{k=1}^m u_{ik}$ indicates whether we end up at $p_i$.
The constraints ensure we output a path of exactly $l$ POIs;
the last constraint in particular ensures we do not have (disjoint) cycles, as per \citet{Miller:1960}.
By reading off the values of $u_{jk}$, we can determine the predicted sequence.

The above ILP implicitly solves both the problem of selecting the set of $l$ POIs to recommend,
and the problem of ordering them.
We could fix the set of POIs to recommend, \eg by using the POIs in the Viterbi solution, and then find the optimal ordering of these nodes.
However, we have found such an approach to have only modest improvement over the loop elimination heuristic of the previous section, which also builds upon the Viterbi solution.
