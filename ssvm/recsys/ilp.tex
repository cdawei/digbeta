% !TEX root=main.tex

% \blue{
% We should probably ask Phil Kilby or Hassan about the two or three most popular subtour eliminations

% \begin{itemize}
% 	\item Elementary shortest path problems
%   %http://www.optimization-online.org/DB_FILE/2014/09/4560.pdf
% 	\item Different subtour eliminations
%   	\item (Dantzig, Fulkerson, Johnson, 1954) has 3 different versions
%   	\item (Miller, Tucker, Zemlin, 1960) has another one.
%   	\item Seems to be one more, not sure by whom (maybe Belmore, Malone, 1971)
%   %http://www.or.unc.edu/~pataki/papers/teachtsp.pdf

% \end{itemize}
% }

Instead of simply bypassing repeated revisits, 
we can make use of subtour elimination techniques developped by the TSP research community,
by formulating the trip recommendation problem with integer linear programming (ILP).
The additional benefits of this formulation is that fixing the subset of POIs is not required any more,
we can optimise over all possible subset of POIs using the off-the-shelf ILP solvers.
In particular, given a starting location $s$ and the required trip length $l$,
we maximise the desired trip score over all possible subsets with $l$ POIs from the whole set of POIs $\{p_j\}_{j=1}^m$, \ie

\begin{align*}
\max_{\bu} & \sum_{k=1}^m \w_k^\top \psi_k(\x, p_k) \sum_{j=1}^m u_{jk} +
             \sum_{j,k=1}^m u_{jk} \w_{jk}^\top \psi_{j, k}(\x, p_j, p_k) \\
s.t. 
& \sum_{k=2}^m u_{1k} = 1, \, \sum_{j=2}^m u_{j1} = z_1=0  \\                %\label{eq:cons1} \\
& \sum_{j=1}^m \sum_{k=1}^m u_{jk} = l-1, \, \sum_{j=1}^m u_{jj}=0  \\      %\label{eq:cons2}
& \sum_{j=1}^m u_{ji} = z_i + \sum_{k=2}^m u_{ik} \le 1, \, (\forall i \in \{2,\cdots,m\})  \\ %\label{eq:cons3} \\
& v_j - v_k + 1 \le (m-1) (1-u_{jk}), \, (\forall j,k \in \{2,\cdots,m\})                 %\label{eq:cons4}
\end{align*}
where $u_{jk}$ are binary variables that are true if and only if
we visit $p_k$ immediately after visiting $p_j$,
and binary variables $z_j$ indicates whether we end up at $p_j$.
Note that we index POIs such that $s = p_1$ for brevity,
and we strictly ask for a path that starts from $s$ and travels $l$ distinct POIs.
