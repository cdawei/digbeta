% !TEX root=main.tex

A burgeoning sub-field of citizen-centric recommendation focusses on suggesting travel routes in a city that a tourist might enjoy.
This goal encompasses at least three distinct problems:
\begin{enumerate}[(1)]
	\item ranking \emph{all} points of interest (POIs) in a city in a manner personalised to a tourist (\eg a tourist interested in scenic views might have {\tt Opera House $\succ$ Darling Harbour $\succ$ The Rocks}) \citep{shi2011personalized,lian2014geomf,hsieh2014mining,yuan2014graph};
	\item recommending the \emph{next} location a tourist might enjoy, given the sequence of places they have visited thus far (\eg given {\tt The Rocks$\to$Botanic Gardens}, we might recommend {\tt Hyde Park}) \citep{fpmc10,ijcai13,zhang2015location};
	%and
	\item recommending an \emph{entire sequence} of POIs for a tourist, effectively giving them a travel itinerary (\eg {\tt Opera House$\to$Darling Harbour$\to$Chinatown}) \citep{lu2010photo2trip,ijcai15,lu2012personalized,gioniswsdm14,chen2015tripplanner}.
\end{enumerate}
Our focus in the present paper is problem setting (3), which we dub ``trajectory recommendation''.

%There are at least two challenges in effectively tackling trajectory recommendation.
%First,
Effectively tackling trajectory recommendation poses a challenge:
at training time, how can one design a model that can recommend sequences of POIs which are coherent as a \emph{whole}?
Merely concatenating a tourists' personalised top ranking POIs into a sequence
might result in prohibitive travel (e.g.\ {\tt Opera House$\to$Royal National Park}),
or may suffer from a lack of diversity (e.g.\ we might recommend three restaurants in a row).
This motivates an inherently structured approach to the problem.
In recent work, \citet{Chen:2017} explored the use of structured SVMs as one such approach, and showed that this outperforms POI ranking approaches. %as well as Markov model approaches.

In this paper, we study a distinct but related challenge:
at prediction time, how can one recommend a sequence that does not have \emph{loops}?
This is desirable because tourists would typically wish to avoid revisiting a POI that has already been visited before.
In principle, such a problem will not exist if one employs a suitably powerful model during training.
In practice, one is forced to compromise on model richness owing to computational and sample complexity considerations.
%As a result, it is of import to design means of overcoming this problem.

How can one solve this modified inference problem?
We study this question with the following contributions:
\begin{enumerate}
	\item[(\textbf{C1})] We detail three different approaches to the problem -- 
	%a heuristic inspired by Christofides' algorithm,
	graph-based heuristics,
	list extensions of the Viterbi algorithm, and integer linear programming
	-- and qualitatively summarise their strengths and weaknesses.
	\item[(\textbf{C2})] In the course of our analysis, we explicate how two ostensibly different approaches to the list Viterbi algorithm \citep{seshadri1994list,nilsson2001sequentially} are in fact fundamentally identical.
	\item[(\textbf{C3})] We conduct experiments on real-world trajectory recommendation datasets to identify the tradeoffs imposed by each of the three approaches.
\end{enumerate}
Overall, we find that
all methods offer performance improvements over na\"{i}vely predicting a sequence with loops;
Christofides' algorithm is at least an order of magnitude faster than the other methods, but with a significant sacrifice in accuracy;
and that the list Viterbi is faster than the ILP for short trajectories, but the ILP is superior for longer trajectories.

%The paper is organised as follows:
% In the sequel,
% Section \ref{sec:background} formalises the problem setting;
% Sections \ref{sec:christofides} -- \ref{sec:viterbi} summarise the three distinct approaches;
% Section \ref{sec:experiments} provides empirical comparison of the methods;
% and Section \ref{sec:discussion} gives some additional discussion and directions for future research.
