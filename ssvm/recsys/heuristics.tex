% !TEX root=main.tex

To design approximations, it will be useful to view Equation \ref{eqn:argmax-path} as a graph optimisation problem
for the case of structured SVMs with pairwise potentials.
Here, the score $f(\mathbf{x}, \mathbf{y})$ can be decomposed as
%$$ f(\mathbf{x}, \mathbf{y}) = \sum_{k = 1}^{|\mathbf{y}|} \mathbf{w}_k^T \Phi( \mathbf{x}, y_k ) + \sum_{j, k = 1}^{|\mathbf{y}|} \mathbf{w}_{j k}^T \Phi( \mathbf{x}, y_k, y_j ) $$
%for unary and pairwise weight vectors $\mathbf{w}_k$.
\begin{equation}
	\label{eqn:f-ssvm}
	f(\mathbf{x}, \mathbf{y}) = \sum_{k = 1}^{|\mathbf{y}|} \alpha( y_k ) + \sum_{j, k = 1}^{|\mathbf{y}|} \beta( y_k, y_j )
\end{equation}
for suitable unary and pairwise potentials $\alpha, \beta$.
Consequently, we can imagine a complete graph where the nodes are POIs, each node $p \in \PCal$ has a score $\alpha( p )$, and each edge $(p, p') \in \PCal^2$ has a score $\beta( p, p' )$.
Then, Equation \ref{eqn:argmax-path} is equivalently a problem of selecting $l$ nodes such that
the total sum of node and edge scores in the selection maximises Equation \ref{eqn:f-ssvm}.
This points to a natural means of approximation: solve this selection problem greedily, or fix the set of nodes and only solve the ordering problem.
%the following selection problem:
%$$	r( x ) = \argmax_{\mathbf{y} \in \thickbar{\YCal}_x}~\sum_{k = 1}^{|\mathbf{y}|} \alpha( y_k ) + \sum_{j, k = 1}^{|\mathbf{y}|} \beta( y_k, y_j ). $$

\subsection{Greedy path discovery}
\label{sec:greedy}
% !TEX root=main.tex

In light of the graph-based view, a natural approach to approximately solve Equation \ref{eqn:argmax-path} is a greedy algorithm.
Suppose we have already determined a partial path comprising distinct POIs $y_1, \ldots, y_k$.
Then, we can select the next candidate POI $y_{k + 1}$ as
the node
that iteratively optimises Equation \ref{eqn:f-ssvm},
%whose combined node- and edge-score is maximum,
subject to the constraint that it is distinct from all other nodes in the current path;
formally, we pick
%the node whose combined node score $\alpha( y_{k + 1} )$ and
%transition score $\beta( y_{k}, y_{k + 1} )$
$$ y_{k + 1} = \argmax_{p \in \PCal - \{ y_1, \ldots, y_k \}} \alpha( p ) + \beta( y_k, p ). $$
This algorithm is faster that the above heuristic, with $\mathscr{O}(l \cdot | \PCal |)$ complexity,
but similarly has an unclear performance guarantee.


\subsection{Heuristic loop elimination}
\label{sec:loop-elim}
% !TEX root=main.tex

An even simpler approach than the above greedy strategy is the following:
by default, return the standard Viterbi solution (Equation \ref{eqn:argmax});
if this solution has loops, then simply remove them.
Specifically, if the Viterbi solution is $( y_1, \ldots, y_l )$,
and $i$ is the first index where $y_i$ already appears in the subsequence $( y_1, \ldots, y_{i-1} )$,
then simply return the subsequence $( y_1, \ldots, y_{i-1} )$.

The above algorithm is guaranteed to return a path,
but has at least two detrimental features.
First, it returns a solution that violates the length constraint of the trajectory recommendation problem.
Second, if we restrict attention to the POIs $\{ y_1, \ldots, y_{i-1} \}$, it is unclear whether the original ordering of the subsequence is optimal.
The second point can be remedied, as we now see.


% \subsection{Heuristic subtour elimination}
% \label{sec:christofides}
% % !TEX root=main.tex

%\blue{How to use results from travelling salesman for trajectory recommendation?} 

%\noindent
The problem of recommending a trip over a set of POIs can be reduced to the classic travelling salesman problem (TSP)
if every POI is restricted to visit once.
However, in practice, we normally only visit a subset of these POIs,
which means results of TSP cannot be trivially used unless the subset of POIs is fixed.

%\noindent
%\blue{Not obvious how to directly use Christofides? Do we just find double visits and bypass?} 
%https://research.googleblog.com/2016/09/the-280-year-old-algorithm-inside.html

%\noindent
One well-known heuristic to approximately solve the TSP is the Christofides algorithm \citep{Christofides:1976}.
This algorithm finds a circuit that cost at most 1.5 times
of the optimal TSP solution, by constructing a minimum spanning tree and matching certain nodes, 
building the solution by simply bypassing repeated nodes.

Inspired by this, and recalling that the recommended trips by the classic Viterbi algorithm cannot avoid repeated visits,
we can first request a longer sequence using Viterbi and then skip repeated visits to form a trip, 
we keep asking for sequence with different length, until we cannot improve the resulting trip (with respect to the required length).
This algorithm is denoted as \textsc{Heuristic} in experiment.

