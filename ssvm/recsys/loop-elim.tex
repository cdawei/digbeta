% !TEX root=main.tex

%An even simpler approach than the above greedy strategy is the following:
Perhaps the simplest approximate solution to Equation \ref{eqn:argmax-path} is to merely remove the first loop occurring in the standard Viterbi solution (Equation \ref{eqn:argmax}).
That is, if the Viterbi solution is $( y_1, \ldots, y_l )$,
return the sub-sequence $( y_1, \ldots, y_{i-1} )$
for the first index $i$ where $y_i$ appears in this sub-sequence.
This has complexity dominated by the Viterbi algorithm, \emph{viz.} $\mathscr{O}(l \cdot |\PCal|^2)$ for an input query $\x = (s, l)$.

From a graph perspective, this approach makes the directed sub-graph induced by the Viterbi solution acyclic
by breaking the first cycle-inducing edge.
This is sensible if sequences never escape the first cycle, \ie after the first repeated POI, there is no new POI.
We have indeed found this to be the case in our problem; %(perhaps owing to cycles being induced by dominant edge scores).
more generally, %for sequences %such as $(1,2,1,3)$
%where there is a repeated POI followed by a new POI,
the problem of removing cycles is a special case of the ({\sf NP}-hard) minimum feedback arc-set problem \citep[pg.\ 192]{Garey:1990}.

This algorithm is appealing in its simplicity,
but has at least two drawbacks.
First, it makes the questionable assumption that Equation \ref{eqn:argmax-path} is solvable from the standard Viterbi solution alone.
Second, the solution violates the length constraint of the path recommendation problem.
As a remedy,
we can request the Viterbi algorithm to return a sequence of longer length
%$l' > l$;
%however, there is no clear means of choosing $l'$ so that the result is exactly of length $l$.
%We can thus run the algorithm for each
$l' \in \{ l, l + 1, \ldots, |\PCal| \}$, and pick the (smallest) $l'$ for which the predicted length is closest to $l$.

% Second, if we restrict attention to the POIs $\{ y_1, \ldots, y_{i-1} \}$, it is unclear whether the original ordering of the subsequence is optimal.
% The second point can be remedied, as we now see.
