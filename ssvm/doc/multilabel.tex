\documentclass[9pt]{extarticle}
\usepackage[a4paper,top=0.79in,left=0.79in,bottom=0.79in,right=0.79in]{geometry} % A4 paper margins in LibreOffice
\usepackage{hyperref}
\usepackage{amsmath}
\usepackage{amsthm}
\usepackage{amsfonts}
\usepackage{mathrsfs}
\usepackage{bm}
\usepackage{bbm}
%\usepackage{ulem}
\usepackage{stmaryrd}
\usepackage{algorithm}
\usepackage{algorithmic}
\usepackage[sc]{mathpazo}
\linespread{1.05}       % Palladio needs more leading (space between lines)
\usepackage[T1]{fontenc}
\usepackage{footmisc}   % \footref, refer the same footnote at different places
\usepackage{subcaption} % sub-figures
\usepackage{setspace}   % set space between lines
\usepackage[utf8]{inputenc}
\usepackage[english]{babel}
\usepackage{xcolor}
\usepackage{graphicx}
\graphicspath{{fig/}}   % Location of the graphics files

\newtheorem{theorem}{Theorem}
\newtheorem{corollary}{Corollary}
\newtheorem{lemma}{Lemma}

\DeclareMathOperator*{\argmin}{argmin}
\DeclareMathOperator*{\argmax}{argmax}
\newcommand{\eat}[1]{}
\newcommand{\given}{\mid}
\newcommand{\llb}{\llbracket}
\newcommand{\rrb}{\rrbracket}
\newcommand{\bu}{\mathbf{u}}
\newcommand{\bv}{\mathbf{v}}
\newcommand{\h}{\mathbf{h}}
\newcommand{\x}{\mathbf{x}}
\newcommand{\X}{\mathbf{X}}
\newcommand{\Y}{\mathbf{Y}}
\newcommand{\y}{\mathbf{y}}
\newcommand{\z}{\mathbf{z}}
\newcommand{\1}{\mathbf{1}}
\newcommand{\w}{\mathbf{w}}
\newcommand{\p}{\mathbb{P}}
\newcommand{\E}{\mathbb{E}}
\newcommand{\R}{\mathbb{R}}
\newcommand{\Z}{\mathbb{Z}}
\newcommand{\q}{\mathbf{q}}
\newcommand{\LCal}{\mathcal{L}}
\newcommand{\XCal}{\mathcal{X}}
\newcommand{\YCal}{\mathcal{Y}}
\newcommand{\alphat}{\tilde{\alpha}}
\newcommand{\betat}{\tilde{\beta}}
\newcommand{\gammat}{\tilde{\gamma}}
\newcommand{\phit}{\tilde{\phi}}
\newcommand{\alphabm}{\bm{\alpha}}
\newcommand{\betabm}{\bm{\beta}}
\newcommand{\nubm}{\bm{\nu}}
% madeness: suPer-script in Brackets
\newcommand{\pb}[1]{^{({#1})}}

\newcommand{\eg}{e.g.\ }
\newcommand{\ie}{i.e.\ }
\newcommand{\downto}{\,\textbf{downto}\,}
\newcommand{\blue}[1]{{\color{blue}{#1}}}

\setlength{\columnsep}{1.5em} % spacing between columns

\title{Multi-label Classification, Bipartite Ranking and Playlist Generation}

\author{Dawei Chen}

\date{\today}

\begin{document}

\maketitle

\section{Multi-label classification~\cite{cheng:2010}}
\label{sec:mlc}

%1. Brief summary of reference

\noindent
\paragraph{Definition}
Let $\LCal = \{\lambda_1,\dots,\lambda_l\}$ be a finite set of class labels,
and example $(\x,\y) \in \XCal \times \YCal$, 
where $\YCal \in \{0,1\}^m$ is the set of all possible labels,
and $\y=y_{1:m}$ is a binary vector where $y_i = 1$ \emph{iff} $\lambda_i$ is a label of $\x$.
A multi-label classifier is a mapping $\h: \XCal \to \YCal$.

\noindent
\paragraph{Label dependence}
Suppose examples are independent and identically distributed (iid) according to a joint probability distribution $\p(\X,\Y)$ on $\XCal \times \YCal$,
where $\X$ is a random variable and $\Y=Y_{1:l}$ is a random vector,
Let $\p\pb{i}(Y_i |\x)$ be the marginal distribution of $Y_i$, then
\begin{equation*}
\p\pb{i}(Y_i=b |\x) = \sum_{\y \in \YCal:y_i = b} \p(\Y = \y |\x),
\end{equation*}
where $\p(\Y = \y |\x)$ is the posterior distribution given observation $\x$.
We note that the labels are not independent if 
\begin{equation*}
\p(\Y |\x) \ne \prod_{i=1}^l \p\pb{i}(Y_i |\x),
\end{equation*}
and the degree of dependence could be quantified in terms of measures such as cross entropy and KL divergence.

\noindent
\paragraph{Learning}
Given a loss function $\ell(\cdot)$, 
we can learn a multi-label classifier by find a model $\h^*$ that minimise the expected loss over the joint distribution $\p(\X,\Y)$:
\begin{equation*}
\h^* 
= \argmin_{\h} \, \E_{\X\Y} \, \ell(\Y,\h(\X))
= \argmin_{\h} \, \E_{\X} \, \E_{\Y|\X} \, \ell(\Y,\h(X))
= \argmin_{\h} \, \sum_{\x} \p(\x) \, \E_{\Y|\X} \, \ell(\Y,\h(\x)),
\end{equation*}
thanks to the summation, fix $\x$, we have
\begin{equation*}
\h^*(\x) = \argmin_{\y} \, \E_{\Y|\X} \, \ell(\Y,\y).
\end{equation*}
Frequently used loss functions in the context of multi-label classification including Hamming loss, rank loss and subset 0/1 loss~\cite{cheng:2010},
here we focus on a rank loss (taking care of ties):
\begin{equation}
\label{eq:loss_rank}
\ell(\y, \h(\x)) = \sum_{(i,j): y_i > y_j} \left( \llb h_i < h_j \rrb + \frac{1}{2} \llb h_i = h_j \rrb \right).
\end{equation}
\emph{Theorem 3.1 in~\cite{cheng:2010} here.}

\noindent
\paragraph{Probabilistic classifier chains}
Given a query $\x$, the posterior probability of a label $\y$ can be computed using the product rule of probability:
\begin{equation*}
\p(\y |\x) = \p(y_1) \cdot \prod_{i=2}^l \p(y_i |\x, y_{1:i-1}),
\end{equation*}
and we further define a function:
\begin{equation*}
f_i = 
\begin{cases}
\p(y_i = 1 |\x), & i = 1 \\
\p(y_i = 1 |\x, y_{1:i-1}), & 1 < i \le l
\end{cases}
\end{equation*}
then we have
\begin{equation*}
\p(\y |\x) = f_1 \cdot \prod_{i=2}^l f_i,
\end{equation*}
where $f_i$ uses $\x$ and $y_{1:i-1}$ as the input features. 
Theoretically, the results of the product rule does not depend on the order of variables, 
however, in practice, different order of variables will result in different model parameters (\ie the order of features depend on the order of variables). \\
\emph{Greedy approach -- classifier chain; assuming Markov property, we can use the Viterbi algorithm; with Neural net, we can build an order agnostic model.}


\section{Bipartite ranking}
\label{sec:birank}

%1. Brief summary of reference

\paragraph{Definition} 
Bipartite ranking is to learn a real-valued ranking function that places positive examples above negative examples~\cite{li:2014}.
Formally, given training examples $S = S_+ \cup S_-$ with $m$ positive examples $S_+ = \{\x_i^+\}_{i=1}^m$ and $n$ negative examples $S_- = \{\x_i^-\}_{i=1}^n$, 
bipartite ranking aims to learn a ranking function $f: \XCal \to \R$ that is likely ranks positive examples higher than negative examples.

\paragraph{Loss function}
AUC is a widely used as an evaluate metric for bipartite ranking, and it turns out that AUC can be optimised by minimising a loss defined as~\cite{cortes:2004}
\begin{equation}
\label{eq:loss_auc}
\ell_\text{rank}(f; S) = \frac{1}{mn} \sum_{i=1}^m \sum_{j=1}^n \llb f(\x_i^+) \le f(\x_j^-) \rrb,
\end{equation}
and this loss can be easily optimised (\eg by gradient descent) if we replace the indicator function with a convex surrogate such as the truncated quadratic loss 
$\ell(z) = (1+z)_+^2$, the exponential loss $\ell(z) = e^z$ and logistic loss $\ell(z) = \log(1+e^z)$.
One drawback of this loss function is enumerating all the positive-negative pairs, which is computationally expensive for large dataset. \\
\emph{Theorem 3.1 in~\cite{cheng:2010} for this loss function here.}

Alternatively, one may interested in optimising the ranking accuracy only at the top, 
or equivalently, we would like to minimize the number of positive examples that ranked below the highest-ranking negative instance~\cite{agarwal:2011,li:2014}:
\begin{equation}
\label{eq:loss_inf}
\begin{aligned}
\ell_{\infty}(f; S) 
&= \max_{1 \le j \le n} \frac{1}{m} \sum_{i=1}^m \, \llb f(\x_i^+) < f(\x_j^-) \rrb \\
&= \frac{1}{m} \sum_{i=1}^m \max_{1 \le j \le n} \llb f(\x_i^+) < f(\x_j^-) \rrb,
\end{aligned}
\end{equation}
by replace the indicator function in (\ref{eq:loss_inf}) with a convex surrogate $\ell(\cdot)$, we have
\begin{equation}
\label{eq:loss_inf1} 
\begin{aligned}
\tilde{\ell}_{\infty}(f; S) 
&= \frac{1}{m} \sum_{i=1}^m \max_{1 \le j \le n} \ell\left( f(\x_j^-) - f(\x_i^+) \right) \\
&= \frac{1}{m} \sum_{i=1}^m \ell\left( \max_{1 \le j \le n} f(\x_j^-) - f(\x_i^+) \right),
\end{aligned}
\end{equation}
which can be optimised more efficiently than (\ref{eq:loss_auc})~\cite{li:2014}.

\paragraph{Dual formulation}
Consider a linear ranking function $f(\x) = \w^\top \x$ and loss function~\ref{eq:loss_inf1}.
Table~\ref{tab:symbol} summarises some notation we will use.
\begin{table}[!h]
\caption{Glossary of commonly used symbols}
\label{tab:symbol}
\renewcommand{\arraystretch}{1.5} % tweak the space between rows
\setlength{\tabcolsep}{1pt} % tweak the space between columns
\centering
\begin{tabular}{llll}
\hline \hline
\multicolumn{3}{l}{\textbf{Symbol}} & \textbf{Quantity} \\ \hline 
$d$              &  $\in$  &  $\Z^+$  & The number of features for each example \\
$\w$             &  $\in$  &  $\R^d$  & The vector of model parameters \\
$\mathbf{1}_m$   &  $\in$  &  $\R^m$  & The $m$ dimensional vector of $1$'s \\
$\X^+$           &  $\in$  &  $\R^{m \times d}\quad$  & Matrix of features of positive examples \\
$\X^-$           &  $\in$  &  $\R^{n \times d}$       & Matrix of features of negative examples \\
$\alphabm$       &  $\in$  &  $\R^m$  &  Dual variables for positive examples \\
$\betabm, \nubm$ &  $\in$  &  $\R^n$  &  Dual variables for negative examples \\ \hline
\end{tabular}
\end{table}

We can learn the model parameters $\w$ by risk minimisation with L2 regularisation:
\begin{equation}
\label{eq:minrisk}
\min_{\w} \, \frac{\lambda}{2} \w^\top \w + \frac{1}{m} \sum_{i=1}^m \ell\left( \max_{1 \le j \le n} \w^\top \x_j^- - \w^\top \x_i^+ \right),
\end{equation}
where $\lambda > 0$ is a regularisation constant.
Problem (\ref{eq:minrisk}) is hard to optimise in general due to the maximum term in loss function, one widely used trick is to form its dual problem.
Let 
\begin{equation*}
\begin{aligned}
f_0 (\w, \xi) &= \frac{\lambda}{2} \w^\top \w + \frac{1}{m} \sum_{i=1}^m \ell\left( \xi - \w^\top \x_i^+ \right), \\
f_j (\w, \xi) &= \w^\top \x_j^- - \xi, \ j \in \{1,\dots,n\}.
\end{aligned}
\end{equation*}
Then problem (\ref{eq:minrisk}) is equivalent to
\begin{equation}
\label{eq:minrisk_lg}
\begin{aligned}
\min_{\w, \xi} \quad & f_0 (\w, \xi) \\
s.t. \quad & f_j (\w, \xi) \le 0, \ j \in \{1,\dots,n\}.
\end{aligned}
\end{equation}
The \emph{Lagrangian} of (\ref{eq:minrisk_lg}) is
\begin{align*}
L(\w, \xi, \nubm) 
&= f_0 (\w, \xi) + \sum_{j=1}^n \nu_j \cdot f_j(\w, \xi) \\
&= \frac{\lambda}{2} \w^\top \w + \frac{1}{m} \sum_{i=1}^m \ell\left( \xi - \w^\top \x_i^+ \right) + \sum_{j=1}^n \nu_j \cdot \left( \w^\top \x_j^- - \xi \right)
\end{align*}
Note that the conjugate of the conjugate of a convex function is itself, \eg $f(\z) = f^{**}(\z) = \sup_{\y} \left( \z^\top \y - f^*(\y) \right)$, we have
\begin{align*}
\frac{1}{m} \sum_{i=1}^m \ell\left( \xi - \w^\top \x_i^+ \right)
&= \frac{1}{m} \sum_{i=1}^m \sup_{\alpha_i} \left( (\xi - \w^\top \x_i^+) \cdot \alpha_i - \ell^*(\alpha_i) \right) \\
&= \sup_{\alphabm} \left[ \frac{1}{m} \sum_{i=1}^m (\xi - \w^\top \x_i^+) \cdot \alpha_i - \frac{1}{m} \sum_{i=1}^m \ell^*(\alpha_i) \right] \\
&= \sup_{\alphabm} \left[ \frac{\xi}{m} \1_m^\top \alphabm - \frac{1}{m} \alphabm^\top \X^+ \w - \frac{1}{m} \sum_{i=1}^m \ell^*(\alpha_i) \right] \\
\end{align*}
where $\ell^*(\cdot)$ is the conjugate of $\ell(\cdot)$.
Further, 
$$\sum_{j=1}^n \nu_j \cdot \left( \w^\top \x_j^- - \xi \right) = \nubm^\top \X^- \w - \xi \1_n^\top \nubm$$
Then
\begin{align*}
L(\w, \xi, \alphabm, \nubm) 
&= \frac{\lambda}{2} \w^\top \w + 
   \sup_{\alphabm} \left[ \frac{\xi}{m} \1_m^\top \alphabm - \frac{1}{m} \alphabm^\top \X^+ \w - \frac{1}{m} \sum_{i=1}^m \ell^*(\alpha_i) \right] +
   \nubm^\top \X^- \w - \xi \1_n^\top \nubm \\
&= \sup_{\alphabm} \left[ 
   \frac{\lambda}{2} \w^\top \w + 
   \frac{\xi}{m} \1_m^\top \alphabm - \frac{1}{m} \alphabm^\top \X^+ \w - \frac{1}{m} \sum_{i=1}^m \ell^*(\alpha_i) +
   \nubm^\top \X^- \w - \xi \1_n^\top \nubm \right] \\
&= \sup_{\alphabm} \left[ g(\w, \xi) - \frac{1}{m} \sum_{i=1}^m \ell^*(\alpha_i) \right]
\end{align*}
where
$$g(\w, \xi) = \frac{\lambda}{2} \w^\top \w + \frac{\xi}{m} \1_m^\top \alphabm - \frac{1}{m} \alphabm^\top \X^+ \w + \nubm^\top \X^- \w - \xi \1_n^\top \nubm$$
The \emph{Lagrangian dual function} of (\ref{eq:minrisk_lg}) is
\begin{align*}
\inf_{\w, \xi} \, L(\w, \xi, \alphabm, \nubm) 
&= \inf_{\w, \xi}  \, \sup_{\alphabm} \left[ g(\w, \xi) - \frac{1}{m} \sum_{i=1}^m \ell^*(\alpha_i) \right] \\
&= \sup_{\alphabm} \, \inf_{\w, \xi} \left[ g(\w, \xi) - \frac{1}{m} \sum_{i=1}^m \ell^*(\alpha_i) \right] ~~ \text{(assuming strong duality?)} \\
&= \max_{\alphabm} \, \min_{\w, \xi} \left[ g(\w, \xi) - \frac{1}{m} \sum_{i=1}^m \ell^*(\alpha_i) \right] \\
&= \max_{\alphabm} \left[ \min_{\w, \xi} g(\w, \xi) - \frac{1}{m} \sum_{i=1}^m \ell^*(\alpha_i) \right]
\end{align*}
To solve the unconstrained inner minimisation, let
\begin{align*}
\frac{\partial g}{\partial \w}  &= \lambda \w - \frac{1}{m} \alphabm^\top \X^+ + \nubm^\top \X^- = 0 \\
\frac{\partial g}{\partial \xi} &= \frac{1}{m} \1_m^\top \alphabm - \1_n^\top \nubm = 0
\end{align*}
Then we have
$$
\w^* 
= \frac{1}{\lambda m} \left( \alphabm^\top \X^+ - m \nubm^\top \X^- \right) 
= \frac{1}{\lambda m} \left( \alphabm^\top \X^+ - \betabm^\top \X^- \right) 
$$
and
$$\1_m^\top \alphabm = m \1_n^\top \nubm = \1_n^\top \betabm$$
where $\betabm = m \nubm$, and
\begin{align*}
\min_{\w, \xi} g(\w, \xi) 
&= \frac{\lambda}{2} \w^{*\top} \w^* + \frac{\xi}{m} \1_m^\top \alphabm - \frac{1}{m} \alphabm^\top \X^+ \w^* + \nubm^\top \X^- \w^* - \xi \1_n^\top \nubm \\
&= \frac{\lambda}{2} \w^{*\top} \w^* + 
   \frac{\xi}{m} \left( \1_m^\top \alphabm - m \1_n^\top \nubm \right) - 
   \frac{1}{m} \left( \alphabm^\top \X^+ - m \nubm^\top \X^- \right)^\top \w^* \\
&= \frac{\lambda}{2} \w^{*\top} \w^* + 0 - \lambda \w^{*\top} \w^* \\
&= -\frac{\lambda}{2} \w^{*\top} \w^* \\
&= -\frac{1}{2 \lambda m^2} \left\| \alphabm^\top \X^+ - \betabm^\top \X^- \right\|^2
\end{align*}
Lastly, the \emph{Lagrangian dual problem} of (\ref{eq:minrisk_lg}) is
\begin{align*}
\max_{\nubm} \, \inf_{\w, \xi} \, L(\w, \xi, \alphabm, \nubm) 
= \max_{\alphabm, \nubm} \, \left[ \min_{\w, \xi} g(\w, \xi) - \frac{1}{m} \sum_{i=1}^m \ell^*(\alpha_i) \right]
= \max_{\alphabm, \nubm} \, \left[ -\frac{1}{2 \lambda m^2} \left\| \alphabm^\top \X^+ - \betabm^\top \X^- \right\|^2 - \frac{1}{m} \sum_{i=1}^m \ell^*(\alpha_i) \right]
\end{align*}
subject to $\1_m^\top \alphabm = m \1_n^\top \nubm$,
which is equivalent to
\begin{equation}
\label{eq:minrisk_dual}
\begin{aligned}
\min_{\alphabm, \betabm} \quad & \frac{1}{2 \lambda m} \left\| \alphabm^\top \X^+ - \betabm^\top \X^- \right\|^2 + \sum_{i=1}^m \ell^*(\alpha_i) \\
s.t. \quad & \1_m^\top \alphabm = \1_n^\top \betabm.
\end{aligned}
\end{equation}


\section{Playlist generation}
\label{sec:playlist}

%2. Formal problem statement (e.g. input, output)

The playlist generation problem can be formulated as a multi-label classification problem (we have a label for each song in library).
The idea is to replace the loss function (\ref{eq:loss_rank}) in multi-label classification with loss~(\ref{eq:loss_auc}) or (\ref{eq:loss_inf}). \\
We can build a probabilistic classifier chains with approximate inference (greedy, Viterbi, Neural net).

%3. Baseline approaches (e.g. song-song matrix factorisation, model each label independently, model label pairs with a number of ranking losses)

%4. Evaluation measure (e.g. Precision@k, Average precision, Reciprocal precision)

\bibliographystyle{ieeetr}
%\bibliographystyle{apalike}
\bibliography{ref_mlc}

\end{document}
