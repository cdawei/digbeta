% !TEX root = ./main.tex

\secmoveup
\section{Conclusion}
\textmoveup

We showed how the problem of sequential recommendation,
which arises in contexts such as
recommending a playlist of songs, or a trajectory of points-of-interest in a city,
can be viewed as a structured prediction problem.
We proposed a structured SVM for this recommendation task, with two important changes:
first, we modified the training objective to account for the existence of multiple ground truths;
second, we modified the prediction and loss augmented inference procedures to avoid predicting loops in the sequence via an extension of the classic Viterbi algorithm.
Experiments on two real-world trajectory recommendation datasets showed the benefits of our approach over existing, non-structured recommendation approaches.

There are several important directions for future work.
First, it is of interest to assess the viability of probabilistic approaches to structured prediction,
such as the conditional random field and maximum entropy Markov model.
Second, extending our approaches to additionally capture latent user- and POI-representations, when sufficient personalised data of this type is available, would be of interest.
Third, applying structured prediction approaches to other sequential prediction problems, such as playlist generation, may indicate similar benefits as observed in trajectory recommendation.
%Third, it will be beneficial to consider loss-augmented inference for more complex performance measures.
